%%%%%%%%%%%%%%%%%%%%%%%%%%%%%%%%%%%%%%%%%%%%%%%%%%%%%%%%%%%%%%%%%
%                    HOW TO PROCESS THIS FILE                   %
%%%%%%%%%%%%%%%%%%%%%%%%%%%%%%%%%%%%%%%%%%%%%%%%%%%%%%%%%%%%%%%%%
%
%  To process these files, it is assumed that you are using the 
%  program Latex2e, the ams-math version for equations.  To do
%  thishere at the Math Dept, in your .cshrc file you need to have 
%
%  /usr/local/TeX/bin
%
%  in your path, and you should 
% 
%  unsetenv TEXINPUTS unsetenv TEXFONTS unsetenv TEXFORMATS 
%
%  if these environment variables are set.
%
%  Once you have this file (swig_diss.tex) made up, you turn 
%  it into a .dvi file (swig_diss.dvi) by typing (at the Unix
%  prompt)
%
%        ``latex swig_diss.tex''  
%
%  (you'll have to do this up to 3 times, in order to create 
%  the files which contain the referencing/numbering info for 
%  citations, equations, section numbers, etc.).  You can view 
%  this .dvi file by typing (at the Unix prompt) 
%
%        ``xdvi main.dvi''
% 
%  which will pop up a window where you can view the final 
%  product without ever wasting a piece of paper. Once you've 
%  proofread your file via xdvi and are ready to print out your
%  final version, you need to turn the document into a postscript 
%  (swig_diss.ps) file by typing (at the Unix prompt)
%
%        ``dvips swig_diss.dvi -o swig_diss.ps''  
%
%  Note that dvips has several options (just type dvips with no 
%  arguments to view them). For instance, to save just pages 10-20 
%  to a postscript file, type (at the Unix prompt)
%
%        ``dvips -p 10 -l 20 swig_diss.dvi -o swig_diss.ps''  
%
%  This creates the postscript file swig_diss.ps which contains just
%  the pages 10-20 of swig_diss.dvi. (If you omit the -o option, then 
%  no .ps file will be created, and the generated postscript info will 
%  go directly to your default printer).
%
%  At this point, the postscript file can still be viewed without 
%  printing anything: at the Unix prompt type
%
%        ``ghostview swig_diss.ps''
%
%  This will pop up a separate window through which you can view
%  your file swig_diss.ps exactly as it would look printed out. I 
%  take every opportunity to avoid wasting paper.  Once you do want 
%  a hard copy, you can type (at the Unix prompt)
%
%        ``lpr swig_diss.ps''
%
%  which will print the postscript file swig_diss.ps at your
%  default printer.  Most of the formatting work will be taken 
%  care of by the beautiful latex2e class file written by Marcel 
%  Oliver (ua-thesis.cls). If desired you can move a copy of 
%  ua-thesis.cls to the directory which contains this file
%  (swig_diss.tex), and edit/modify it however you want. The 
%  copy in your directory will then be the copy which latex2e 
%  references for instructions. You can find the system's copy in 
%
%  /usr/local/TeX/lib/texmf/tex/latex2e/dissertation
%
%  The file which documents this handy class file is called 
%  ua-classes.ps, and is truly a gem (it should be available 
%  with the latex2e info, under UofA dissertation document 
%  classes...thanks, Marcel!).
%
%%%%%%%%%%%%%%%%%%%%%%%%%%%%%%%%%%%%%%%%%%%%%%%%%%%%%%%%%%%%%%%%%
%                    SOME LATEX2E SPECIFICS                     %
%%%%%%%%%%%%%%%%%%%%%%%%%%%%%%%%%%%%%%%%%%%%%%%%%%%%%%%%%%%%%%%%%
%
%  In a latex file (such as the one you are presently viewing),
%  each time the command 
%
%  \input{intro_stuff} 
%
%  is given, it's as if all of the stuff from ``intro_stuff.tex'' 
%  gets substituted right there.  If you'd like to refer back 
%  to the stuff that just got inserted, you have to label it with 
%  the command 
%
%  \label{intro_chpt}
%
%  which allows you to refer back to this chapter with the command 
%
%  \ref{intro_chpt} 
%
%  This command will obviously occur rather frequently in the 
%  files containing the actual stuff. You can see a good amount 
%  of such stuff in the source files referenced by this file 
%  (swig_diss.tex): abstract.tex, acknowledgements.tex, conc.tex, 
%  dedication.tex, intro_hist.tex, intro_models.tex, pde_derv.tex, 
%  pde_gen.tex, pde_motiv.tex, and swig_diss.bbl, mickey.ps
%
%  This is supposed to make your life easier.. . ..feel free 
%  to copy whatevery you'd like, and just fill in blinks, put 
%  in your name/info/files/whatever!...I suggest that you keep 
%  one directory as a reference for something that does work 
%  (again, you should be able to get the final copy just by 
%  entering ``latex swig_diss.tex'' 3 times).  The hope is that 
%  you can simply make a parallel directory containing your 
%  dissertation info, and your own main.tex file (which 
%  should look a heck of a lot like this swig_diss.tex file), 
%  enter those same 3 commands (``latex main.tex'') and not 
%  have a problem in the world...(I'm crossing my fingers).
%  Definitely go through the documentation first, and a 
%  latex2e or ams-latex book never hurts.
%
%
%%%%%%%%%%%%%%%%%%%%%%%%%%%%%%%%%%%%%%%%%%%%%%%%%%%%%%%%%%%%%%%%%%%
%                                                                 %
%                    Dissertation  - Kevin Kremeyer               %
%                                                                 %
%%%%%%%%%%%%%%%%%%%%%%%%%%%%%%%%%%%%%%%%%%%%%%%%%%%%%%%%%%%%%%%%%%%
%                                                                 %
%                            ROOT FILE                            %
%                                                                 %
%%%%%%%%%%%%%%%%%%%%%%%%%%%%%%%%%%%%%%%%%%%%%%%%%%%%%%%%%%%%%%%%%%%
%
%  the [final] option will double space the document...save a tree...
%  don't use it until the final final final pre-draft.
%

%\documentclass[final]{ua-thesis}
\documentclass{article}
\usepackage[dvips]{graphicx}
%\usepackage{amssymb}
% 
%%%%%%%%%%%%%%%%%%%%%%%%%%%%%%%%%%%%%%%%%%%%%%%%%%%%%%%%%%%%%%%%%%%
%                                                                 %
%                            TITLE PAGE                           %
%                                                                 %
%%%%%%%%%%%%%%%%%%%%%%%%%%%%%%%%%%%%%%%%%%%%%%%%%%%%%%%%%%%%%%%%%%%
% 
\title{Experimental and Computational Investigations of Binary Solidification}
\author{Kevin P. Kremeyer}
%\degree{Master of Science}
%\degreeabbrev{M.S.}
\begin{document}

\maketitle

\chapter*{Acknowledgements}
\input{acknowledgements}

\chapter*{Dedication}
\input{dedication}
%
%%%%%%%%%%%%%%%%%%%%%%%%%%%%%%%%%%%%%%%%%%%%%%%%%%%%%%%%%%%%%%%%%%%
%                                                                 %
%                         TABLE OF CONTENTS                       %
%                                                                 %
%%%%%%%%%%%%%%%%%%%%%%%%%%%%%%%%%%%%%%%%%%%%%%%%%%%%%%%%%%%%%%%%%%%
%
\tableofcontents
\listoftables
\listoffigures
%
%%%%%%%%%%%%%%%%%%%%%%%%%%%%%%%%%%%%%%%%%%%%%%%%%%%%%%%%%%%%%%%%%%%
%                                                                 %
%                            ABSTRACT                             %
%                                                                 %
%%%%%%%%%%%%%%%%%%%%%%%%%%%%%%%%%%%%%%%%%%%%%%%%%%%%%%%%%%%%%%%%%%%
% 
%\chapter*{ABSTRACT}
\begin{abstract}
  \input{abstract}
\end{abstract}
%
%%%%%%%%%%%%%%%%%%%%%%%%%%%%%%%%%%%%%%%%%%%%%%%%%%%%%%%%%%%%%%%%%%%
%                                                                 %
%                            CHAPTER ONE                          %
%                                                                 %
%%%%%%%%%%%%%%%%%%%%%%%%%%%%%%%%%%%%%%%%%%%%%%%%%%%%%%%%%%%%%%%%%%%
% 
\chapter{ INTRODUCTION } 
\label{intro} 
\markright{} 

\section{Historical Perspective}
\label{intro_hist}
\input{intro_hist}

\section{An Overview of Solidification Models}
\label{intro_models}
\input{intro_models}
%
%%%%%%%%%%%%%%%%%%%%%%%%%%%%%%%%%%%%%%%%%%%%%%%%%%%%%%%%%%%%%%%%%%%
%                                                                 %
%                           CHAPTER FOUR                          %
%                                                                 %
%%%%%%%%%%%%%%%%%%%%%%%%%%%%%%%%%%%%%%%%%%%%%%%%%%%%%%%%%%%%%%%%%%%

\chapter{ PARTIAL DIFFERENTIAL EQUATIONS }
\label{pde}
\markright{}
\input{pde_gen}

\section{Physical Motivation}
\label{pde_motiv}
\input{pde_motiv}

\section{Derivation and Discussion}
\label{pde_derv}
\input{pde_derv}
%
%%%%%%%%%%%%%%%%%%%%%%%%%%%%%%%%%%%%%%%%%%%%%%%%%%%%%%%%%%%%%%%%%%%
%                                                                 %
%                           CHAPTER FIVE                          %
%                                                                 %
%%%%%%%%%%%%%%%%%%%%%%%%%%%%%%%%%%%%%%%%%%%%%%%%%%%%%%%%%%%%%%%%%%%

\chapter{ FUTURE WORK AND CONCLUSIONS}
\label{GeneralFutureWork}
\markright{}

\section{Conclusions}
\label{conc}
\input{conc}

%%%%%%%%%%%%%%%%%%%%%%%%%%%%%%%%%%%%%%%%%%%%%%%%%%%%%%%%%%%%%%%%%%%
%                                                                 %
%                           EXTRA STUFF                           %
%                                                                 %
%%%%%%%%%%%%%%%%%%%%%%%%%%%%%%%%%%%%%%%%%%%%%%%%%%%%%%%%%%%%%%%%%%%
% 
\chapter{EXTRA STUFF}
\label{extra}
\markright{}

\input{EXTRA}

%%%%%%%%%%%%%%%%%%%%%%%%%%%%%%%%%%%%%%%%%%%%%%%%%%%%%%%%%%%%%%%%%%%
%                                                                 %
%                           BIBLIOGRAPHY                          %
%                                                                 %
%%%%%%%%%%%%%%%%%%%%%%%%%%%%%%%%%%%%%%%%%%%%%%%%%%%%%%%%%%%%%%%%%%%
% 
\bibliography{nbm}
\end{document}