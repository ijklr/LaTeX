\documentclass{article}
\usepackage{graphicx}
\usepackage{amsmath, amsthm, amssymb}


\begin{document}

\begin{titlepage}
\begin{center}
\vspace*{1cm}

\Huge
\textbf{Properties of Pseudocompact Space Condensation}


\vspace{0.5cm}

\vspace{1.5cm}

\textbf{Shaun Cheng}

\vfill

A thesis presented for the degree of\\
      Master of Arts

      \vspace{0.8cm}

      \includegraphics[width=0.4\textwidth]{KUSealGS.jpg}

      \Large
      Department of Mathematics\\
	  University of Kansas\\
	  August 2007

	  \end{center}
	  \end{titlepage}






	  \tableofcontents
	  \newpage
	  \begin{center}
	  \Large
	  \textbf{Introduction}
	  \end{center}
	  \vskip 30pt
	  The existence of a compactification is a characteristic property of a Tychonoff space, and one can reasonably expect that the Stone-$\check{C}$ech compactification, which is the largest of all compactifications, plays an important role in the theory of Tychonoff spaces. 

	  \vskip 15pt
	  In 1960, Tamano showed that a Tychonoff space $X$ is paracompact iff $X\times \beta X$ is normal. This is known is the Tamano's Theorem. Around the same time, Glicksberg showed that for Tychonoff spaces $X,Y$, if $X\times Y$ is pseudocompact, then $\beta (X\times Y)=\beta X\times \beta Y$. In 1997, Buzjakova used Glicksberg's Theorem, among other things, to establish a criterion that a pseudocompact space condenses onto a compact space. 
	  Buzjakova's Theorem states that a Tychonoff, pseudocompact space $X$ condenses onto a compact space if and only if the space $X\times T(|\beta X|^+ +1)$ condenses onto a normal space.
	  \vskip 15pt
	  It is interesting to note that Buzjakova's Theorem can be interpreted as a condensation version of the Tamano's Theorem for the pseudocompact case, and that both theorems demonstrate how some properties of a Tychonoff space $X$ can be characterized by the properties of the Stone-$\check{C}$ech compactification $\beta X$.

	  \vskip 15pt
	  In the following chapters, we will prove Tamano's Theorem, Glicksberg's Theorem, some facts and lemmas used in Buzjakova's Theorem, and finally, the proof of Buzjakova's Theorem. 






	  \newpage





	  \section{Tamano's Theorem}



	  \vskip 20pt

	  In Tamano's 1960 paper, he proved that if $X$ is Tychonoff and $X\times \beta X$ normal, then $X$ is paracompact. The converse was well known before Tamano's paper. The following lemmas will be used to prove the converse, that if $X$ is Tychonoff and paracompact, then $X\times \beta X$ is normal.

	  \vskip 20pt

	  \subsection{Lemma.} The product of a paracompact space with a compact Hausdorff space is paracompact.

	  \vskip 10pt

	  \textbf{Proof.} Let $X$ be paracompact, $Y$ compact, and let $\mathcal{U}$ be an open cover of $X\times Y$. For fixed $x\in X,$ as $\{x\}\times Y$ is compact in $X\times Y$, a finite number of elements of $\mathcal{U}$, say $U_{\alpha_1}^x, \dots , U_{\alpha_{n_x}}^x$, cover $\{x\}\times Y$. Pick an open nhood $V_x$ of $x$ in $X$ such that $V_x\times Y\subseteq \bigcup_{i=1}^{n_x} U_{\alpha_i}^x.$ 
	  \vskip 10pt
	  The sets $V_x$, as $x$ ranges through $X$, form an open cover of $X$. By the paracompactness of $X$, let $\mathcal{V}$ be an open locally finite refinement of the sets $V_x$. For each $V\in \mathcal{V}, V\subseteq V_x$ for some $V_x$. 
	  Let $$\mathcal{W}_V=\left\{(V\times Y)\cap U_{\alpha_i}^x: 1\leq i\leq n_x\right\},$$ and let 

	  $$\mathcal{R}=\bigcup \left\{\mathcal{W}_V: V\in \mathcal{V}\right\}.$$

	  \vskip 5pt

	  Since $\mathcal{W}_V\subseteq \mathcal{U}$ for each $V\in \mathcal{V}$, $\mathcal{R}$ is a refinement of $\mathcal{U}$. For each $x\in X$, $W_V$ is a cover for $\{x\}\times Y$ for some $V\in \mathcal{V}$. Thus, $\mathcal{R}$ is a cover of $X\times Y$. Lastly, $\mathcal{R}$ is locally finite because given $(x,y)\in X\times Y,$ there is a neighborhood $U_x$ of $x$ which meets only finitely many $V$'s in $\mathcal{V}$ because $\mathcal{V}$ is locally finite. Then the neighborhood $U_x\times Y$ of $(x,y)$ can then only meet only finitely many sets of $\mathcal{R}$. Hence, $X\times Y$ is paracompact.  \qed


	  \vskip 40pt


	  \subsection{Lemma.}  Every paracompact space is normal. 

	  \vskip 10pt

	  \textbf{Proof.} We first establish regularity. Suppose $A$ is a closed set in a paracompact space $X$ and $x\notin A$. For each $y\in A$, as $X$ is Hausdorff, we can find an open set $V_y$ containing $y$ such that $x\notin \overline{V_y}$. Then the sets $V_y, y\in A,$ together with the set $X\backslash A$, form an open cover of $X$. Let $\mathcal{W}$ be an open locally finite refinement and let 
	  $$V=\bigcup\left\{ W\in \mathcal{W}: W\cap A\neq \emptyset\right\}.$$
	  \vskip 5pt
	  Then $V$ is an open set containing $A$. Now, 
	  $$\overline{V}=\bigcup\left\{ \overline{W}\in \mathcal{W}: W\cap A\neq \emptyset\right\}$$ holds because:\\
	      \textbf{$\supseteq$:} If $z\in \bigcup\left\{ \overline{W}\in \mathcal{W}: W\cap A\neq \emptyset\right\}$, then $z\in \overline{W}$ for some $W \in \mathcal{W}.$ Since $W\subseteq V$, we have $z\in \overline{W}\subseteq \overline{V}.$\\
	      \textbf{$\subseteq$:} If $z\in \overline{V},$ then there exists a net $\{w_\alpha\}\subseteq V=\bigcup\left\{ W\in \mathcal{W}: W\cap A\neq \emptyset\right\}$ converging to $z$. Since $\left\{W\in \mathcal{W}: W\cap A \neq \emptyset\right\}$ is locally finite, the tail of $\{w_\alpha\}$ must be contained in finitely many $W$'s, say $\{W_1,\dots W_n\}$. So, $z\in \overline{W_1 \cup \cdots \cup W_n}=\overline{W_1}\cup \cdots \cup \overline{W_n}.$ Thus, $z\in \overline{W_k}$ for some $k\in \{1,\dots ,n\}$.
	      \vskip 15pt

	      Since $x\notin \overline{V_y}$ for each $y\in A$ and $\overline{W}\subseteq \overline{V_y}$ for each $T\in \left\{W\in \mathcal{W}: W\cap A\neq \emptyset\right\}, x\notin \overline{T}$ for each $T\in \left\{W\in \mathcal{W}: W\cap A\neq \emptyset\right\}.$ So $x\notin \overline{V}.$ Regularity is established. 


	      \vskip 15pt

	      To establish normality, suppose $A$ and $B$ are disjoint closed sets in $X$. For each $y\in A$, by regularity, 
	      we can find an open set $V_y$ such that $y\in V_y$ and $\overline{V_y}\cap B= \emptyset$. Then proceed exactly as before, we can 
	      produce an open set $V$ such that $A\subseteq V$ and $\overline{V}\cap B=\emptyset.$ Thus $X$ is normal. \qed




	      \vskip 35pt



	      \subsection{Theorem.}
	      (Tamano's Theorem) Let $X$ be a Tychonoff space. Then $X \times \beta X$ is normal iff $X$ is paracompact. 

	      \vskip 10pt

	      \textbf{Proof of $\Leftarrow$:} 
	      Since $X$ is paracompact and $\beta X$ is compact $T_2$, by \textbf{Lemma 1.1}, $X\times \beta X$ is paracompact. 
	      By \textbf{Lemma 1.2}, paracompact implies normal and so $X\times \beta X$ is normal. 

	      \vskip 15pt



\textbf{Proof of $\Rightarrow$: } (Based on Engelking) 
    We will prove a slighty stronger version that for any compactification $cX$ of $X$, if $X \times cX$ is normal, 
    then $X$ is paracompact. 
    \vskip 10pt
    Let $cX$ be a compactification of $X$ such that $X \times cX$ is normal. Let 
    $\left\{U_a: a \in A\right\}$ be an open cover of $X$. We will show that it has an open locally finite refinement.

    \vskip 10pt

    $X$ is a subspace of $cX$, so for each $a \in A$, there exists $V_a$ open in $cX$ such that $V_a \cap X = U_a.$ Let $F=cX\backslash \bigcup \left\{V_a: a\in A\right\}$.  We can assume that $F$ is nonempty. Since if $F=\emptyset$, then $cX=\bigcup \left\{V_a: a\in A\right\}$, and since $cX$ is compact, we can find a finite subcover $\left\{V_{a_i}: 1\leq i \leq n\right\}\subseteq \left\{V_a: a\in A\right\}$. Then, $\left\{U_{a_i}: 1\leq i\leq n\right\}$ is an open locally finite refinement of $\left\{U_a: a \in A\right\}$. 


    \vskip 10pt

    Let $\triangle = \{(x,x): x \in X\}.$
    Both $X \times F$ and $\triangle$ are closed in $X \times cX $. Since $X \times cX $ is normal, by Urysohn's Lemma, there is 
    a continuous function $f$: $X \times cX \rightarrow [0,1]$ with $f[\triangle] \subseteq \{0\}$ and $f[X\times F] \subseteq
    \{1\}$. 

    \vskip 15pt 

    Define $d: X\times X \rightarrow \mathbb{R}$ 



    such that $$d(x,y)=\sup \left\{ \left|f(x,z)-f(y,z)\right|: z\in X\right\}$$ for all $(x,y) \in X \times X$. 
    Now, for all $x,y,w \in X$, we have: 

    \begin{enumerate}
    \item $d(x,x)=\sup_{z\in X} |f(x,z)-f(x,z)|=0$
    \item $d(x,y)=\sup_{z\in X} |f(x,z)-f(y,z)|=\sup_{z\in X} |f(y,z)-f(x,z)|=d(y,x)$
    \item $d(x,w)=\sup_{z\in X} |f(x,z)-f(w,z)|\\
		  =\sup_{z\in X} |f(x,z)-f(y,z)+f(y,z)-f(w,z)|\\
		  \leq \sup_{z\in X} |f(x,z)-f(y,z)|+\sup_{z\in X} |f(y,z)-f(w,z)|\\
		  =d(x,y)+d(y,w)$
		  \end{enumerate}


		  Thus $d$ is a pseudometric on $X$. Let $\tau_d(X)$ denote the set of open sets in the topology induced by the pseudometric $d$. 

		  \vskip 10pt

		  For each $B(x_0, \epsilon) \in \tau_d$, pick any point $x' \in B(x_0, \epsilon)$ . Let $\epsilon' = \epsilon - d(x_0,x').$ 
		  The set $\Gamma=\{ G \times H \subseteq X \times cX : G \times H$ is open in $X \times cX, x'\in G,$ and $diam\left(f\left[G\times H\right]\right)<\epsilon' \}$ is an open cover of $\{x'\} \times cX$. To show this, pick any point $(x',y)\in \{x'\} \times cX$. Let $c=f(x',y)\in [0,1].$ Since $f$ is continuous and the set $E=(c-\frac{\epsilon'}{2}, c+\frac{\epsilon'}{2}) \cap [0,1]$ is open in $[0,1], f^\leftarrow[E]$ must be open in $X\times cX$. Since $f^{\leftarrow}[E]$ is an open set that contains $(x',y)$ in $X\times cX$, there exist $G_b \in \tau(X)$ containing $x'$, and $H_b \in \tau(cX)$ containing $y$ such that $G_b\times H_b\subseteq f^\leftarrow[E]$. Thus $G_b\times H_b$ is an element of $\Gamma$. Since $(x',y)$ was arbitrarily chosen from $\{x'\}\times cX$, we conclude that $\Gamma$ is an open cover of $\{x'\}\times cX$.

		  \vskip 10pt

		  $\Gamma$ being an open cover of $\{x'\}\times cX$ means that $\left\{H : G\times H \in \Gamma\right\}$ is an open cover of $cX$. Since $cX$ is compact, there is an finite subcover $\{ H_i: 1\leq i \leq n\}.$ Corresponding to $\{ H_i: 1\leq i \leq n\}$ is the set $\{ G_i: 1\leq i \leq n\}$. Where for each $i \in \{1\dots n\}$, we have $f[G_i \times H_i] \subseteq (c_i-\frac{\epsilon'}{2}, c_i+\frac{\epsilon'}{2})$ for some $c_i \in (0,1).$ Pick any $z \in cX=\bigcup \left\{H_i: 1\leq i\leq n\right\}.$ For some $1 \leq k \leq n$, $z \in H_k$. Then $f[G_k\times \{z\}] \subseteq f[G_k\times H_k]\subseteq (c_k-\frac{\epsilon'}{2}, c_k+\frac{\epsilon'}{2}).$ 

		  \vskip 10pt

		  Let $S=\bigcap\left\{G_i:1\leq i \leq n\right\}  \subseteq G_k.$ Then 
		  $$f[S\times \{z\}]\subseteq f[G_k \times \{z\}] \subseteq (c_k-\frac{\epsilon'}{2}, c_k+\frac{\epsilon'}{2}).$$ 
		  \vskip 5pt
		  For all $x,y \in S, |f(x,z)-f(y,z)|<\epsilon'$, and because this inequality holds true for all $z \in cX,$ $d(x,y)=\displaystyle{\sup_{z\in cX}} |f(x,z)-f(y,z)|\leq \epsilon'.$ Note that $x' \in \bigcap \left\{G_i: 1\leq i \leq n\right\} = S\in \tau(X)$ and that $d(x,y)\leq \epsilon'$ for all $x,y \in S.$ We have $x' \in S \subseteq B(x',\epsilon') \subseteq B(x_0, \epsilon).$ 

		  \vskip 10pt


		  Thus, for each $x' \in B(x_0,\epsilon)$, we can find $S\in \tau(X)$ such that $x'\in S \subseteq B(x_0,\epsilon).$ So, $B(x_0, \epsilon)\in \tau_d(X)$ and $\tau_d(X) \subseteq \tau(X)$.

		  \vskip 10pt


		  By Stone's Theorem, a pseudo-metrizable space is paracompact. So $X$ is paracompact with respect to the pseudo-metrizable topology $\tau_d$. For an open cover $\left\{B(x,\frac{9}{10}): x\in X\right\}$, there is an open locally finite refinement, $\{W_t : t\in T\}.$ Since $\tau_d(X) \subseteq \tau(X)$, $\{W_t: t\in T\} \subseteq \tau(X)$. 


		  \vskip 10pt
		  Pick any $x_0\in X$ and $x' \in B(x_0,\frac{9}{10})$. We have $f(x_0, x')= |f(x_0,x')-0|=|f(x_0,x')-f(x',x')| \leq \sup_{z\in X}\left|f(x_0,z)-f(x_0,z)\right|= d(x_0,x') < \frac{9}{10}$. 
		  So, $f\left[\{x_0\} \times B(x_0, \frac{9}{10})\right] \subseteq [0,\frac{9}{10})$. By continuity, $f\left[\{x_0\}\times \overline{B(x_0,\frac{9}{10})}^{cX}\right] \subseteq [0,\frac{9}{10}].$ So, $\overline{B(x_0,\frac{9}{10})}^{cX} \cap F =\emptyset$ because $f\left[X\times F\right] \subseteq \{1\}.$

		  \vskip 15pt

		  As $\{W_t:  t\in T\}$ refines $\{B(x_0,\frac{9}{10}): x_0\in X\}$ and $\overline{B(x_0,\frac{9}{10})}^{cX} \cap F =\emptyset$ for all $x_0\in X$, $\overline{W_t}^{cX} \cap F = \emptyset$ for every $t \in T$. Then, for each $t\in T$, 
		  $\overline{W_t}^{cX} \subseteq cX\backslash F = \bigcup \left\{V_a: a\in A\right\}$. Since $\overline{W_t}^{cX}$ is compact in $cX$, there exists a finite subcover $\left\{V_j^t: 1\leq j\leq m_t\right\} \subseteq \{V_a: a \in A\}$ such that $\overline{W_t}^{cX} \subseteq \bigcup \left\{V_j^t: 1\leq j \leq m \right\}$. We have: 
		  $$X\cap \overline{W_t}^{cX} \subseteq X\cap (\bigcup \left\{V_j^t: 1\leq j \leq m\right\}).$$
		  $$\mbox{Thus, }X\cap W_t \subseteq X\cap \overline{W_t}^{cX} \subseteq \bigcup \left\{U_j^t: 1\leq j\leq m\right\}.$$

		  \vskip 20pt

		  The set $\{W_t \cap U_j^t: t\in T, 1 \leq j \leq m_t \} \subseteq \tau(X)$ is the desired locally finite open cover of $X$ which refines $\{U_a: a \in A\}$. Hence $X$ is paracompact.\qed









		  \newpage











		  \begin{center}
		  \textbf{CHAPTER II}
		  \end{center}
		  \vskip 40pt
		  \begin{center}
		  \section{Glicksberg's Theorem}
		  \end{center}

		  \vskip 20pt

		  In this chapter, we will prove Glickberg's Theorem. The full version of Glicksberg's Theorem actually states that if the Cartesian product $\Pi_{s\in S} X_s$ is pseudocompact, then $\beta \left(\Pi_{s\in S} X_s\right) = \Pi_{s\in S} \beta X_s.$ However, since we need only the finite version of it in the proof of Buzjakova's Theorem, we will prove the finite version. 

		  \vskip 20pt
		  We  need some preliminary facts before we prove two important lemmas - \textbf{Lemma 2.6} and \textbf{Lemma 2.7}. The proof of Glicksberg's Theorem follows immediately from these two lemmas.


		  \vskip 25pt

		  \subsection{Fact.}  Let $Y$ be an extension of a space $X$, let $Z$ be a regular space, and let $f:X \rightarrow Z$ be continuous. The following are equivalent: 
		  \begin{enumerate}
		  \item There exists a continuous function $F: Y\rightarrow Z$ such that $F|_X=f.$
		  \item For each $y\in Y,$ the filter $\mathcal{F}_y=\{A\subseteq Z: A\supseteq f[U]$ for some $U\in O^y\}$ converges(where $O^y=\{W\cap X: W$ is open in $Y$ and $y\in W$\}).

		  \end{enumerate}

		  \vskip 15pt


		  \textbf{Proof of 1 $\Rightarrow$ 2: } Suppose $F$ exists and $y\in Y$. We will show $\mathcal{F}_y$ converges to $F(y).$ Let $W$ be an open neighborhood of $F(y)$ in $Z$. By continuity, there is an open neighborhood $U$ of $y$ such that $F[U]\subseteq W,$ where $U$ is open in $Y$. Thus, $f\left[U\cap X\right]=F\left[U\cap X\right] \subseteq W$ and $f\left[U\cap X\right]\in \mathcal{F}_y$. Thus, $\mathcal{F}_y$ converges to $F(y)$.



		  \vskip 20pt


		  \textbf{Proof of 2 $\Rightarrow$ 1: } Suppose for each $y\in Y, \mathcal{F}_y$ converges to some point. As $Z$ is regular and hence Hausdorff, $\mathcal{F}_y$ converges to an unique point which we denote by $F(y)$. Thus, we have just defined a function $F:Y\rightarrow X$.

		  \vskip 10pt

		  If $x\in X$ and $W$ is an open neighborhood of $f(x),$ there is an open set $U$ of $X$ with $x\in U$ and $f[U]\subseteq W.$ 
		  If $V$ is an open set in $Y$ such that $V\cap X =U$, then $f\left[V\cap X\right] \in \mathcal{F}_x$. Thus, $\mathcal{F}_x$ converges to $f(x)$ for all $x\in X$. So, $F(x)=f(x)$ for all $x\in X$, i.e., $F|_X=f$.

		  \vskip 15pt

		  To show $F$ is continuous, let $y\in Y$ and let $W$ be an open neighborhood of $F(y)$. As $Z$ is regular, there is an open subset $V$ of $Z$ such that $F(y)\in V\subseteq \overline{V}^Z \subseteq W.$ Since $\mathcal{F}_y$ converges to $F(y)$, there is an open set $U$ of $Y$ such that $y\in U$ and $f\left[ U\cap X\right]\subseteq V.$

		  \vskip 10pt

		  Let $p\in U.$ We will show that $F(p)\in \overline{V}^Z.$ Let $T$ be an open set of $Z$ containing $F(p)$. From the definition of $\mathcal{F}_p$, there is an open set $R\in \tau(Y)$ containing $p$ such that $R\subseteq U$ and $f\left[R\cap X\right]\subseteq T.$ As $X$ is dense in $Y$, $R\cap X\neq \emptyset$. Hence $f\left[R\cap X\right]\neq \emptyset.$ Since $f\left[U\cap X\right]\subseteq V$ and $R\subseteq U$, we have $f\left[R\cap X\right]\subseteq V$. Thus $f\left[R\cap X\right]\subseteq T\cap V.$ As $T\cap V\neq \emptyset,$ and $T$ was an arbitrary open set containing $F(p)$, $F(p)\in \overline{V}^Z$.

		  \vskip 10pt

		  Since for every $p\in U$, $F(p)\in \overline{V}^Z\subseteq W,$ we conclude that $F[U]\subseteq W.$ Thus $F$ is continuous. \qed










		  \vskip 40pt













		  \subsection{Fact.} Let $Y$ be an extension of a space $X$ and let $Z$ be a regular space. Let $g:Y\rightarrow Z$ be such that for each $y\in Y, g|_{X\cup \{y\}}$ is continuous. Then $g$ is continuous.

		  \vskip 15pt

		  \textbf{Proof: } Let 
		  \begin{center}
		  $O_1^y=\{W\cap X: W$ is open in $Y$ and $y\in W\},$ and 

		  $O_2^y=\{W\cap X: W$ is open in $X\cup \{y\}$ and $y\in W$\}. 
		  \end{center}


		  We have $O_1^y=O_2^y$ because: 

		  \vskip 5pt
		  $(\subseteq:) $ Let $W\cap X \in O_1^y$. $W$ is open in $Y$ and $y\in W.$ Since $X\cup \{y\}$ is the subspace of $Y$, $W\cap (X\cup \{y\})$ is open in $X\cup \{y\}$. Also, $y\in W\cap (X\cup \{y\}).$ So, $W\cap X=\left(W\cap \left(X\cup\{y\}\right)\right)\cap X \in O_2^y$.


		  $(\supseteq:)$ Let $W\cap X\in O_2^y$. Since $W$ is open in $X\cup \{y\},$ there exists $V\in \tau(Y)$ such that $W=V\cap (X\cup \{y\})$. Since $y\in W$ and thus $y\in V$, we have $V\cap X \in O_1^y$. As $W\cap X=V\cap X$, $W\cap X \in O_1^y$. 


		  \vskip 20pt

		  Since $g|_{X\cup \{y\}}$ is continuous, then by \textbf{Fact 2.1}, the filter $\{A\subseteq Z: A \supseteq g|_X [U]$ for some $U\in O_2^y\}$ converges
		  to $g(y)$. Since $O_1^y=O_2^y$, we have $\{A\subseteq Z: A\supseteq g|_X [U]$ for some $U\in O_1^y\}$ converging to $g(y)$ as well. By the other direction of \textbf{Fact 2.1}, $g$ is continuous.\qed













		  \vskip 40pt












		  \subsection{Fact.} The following are equivalent for a dense subspace $S$ of the Tychonoff space $X$: 
		  \begin{enumerate}
		  \item $S$ is $C^*$-embedded in $X$. 
		  \item If $Z_1$ and $Z_2$ are disjoint zero-sets of $S$, then $\overline{Z_1}^X\cap \overline{Z_2}^X=\emptyset$.
		  \end{enumerate}


		  \vskip 10pt

		  \textbf{Proof of 1 $\Rightarrow$ 2: }  By the transitivity of $C^*$-embedding, $S$ is $C^*$-embedded and dense in $\beta X$. Thus $\beta S$ is equivalent to $\beta X$. Let $Z_1$ and $Z_2$ are disjoint zero-sets of $S$. Since disjoint zero-sets in $S$ have disjoint closures in $\beta S \equiv_S \beta X$, $\overline{Z_1}^{\beta X}\cap \overline{Z_2}^{\beta X}=\emptyset$. Hence $\overline{Z_1}^X\cap \overline{Z_2}^X=\emptyset$.


		  \vskip 20pt



		  \textbf{Proof of 2 $\Rightarrow$ 1: } It suffices to show that $S$ is $C^*$-embedded in $S\cup \{p\}$ for each $p\in X\backslash S$ by \textbf{Fact 2.2}.  


		  \vskip 15pt

		  So it remains to be shown that for each $p\in X\backslash S,$ $S$ is $C^*$-embedded in $S\cup \{p\}$. Let $p\in X\backslash S$, and let $C(p)$ denote the collection of closed neighborhoods of $p$ in $S\cup \{p\}.$ If $f\in C^*(S)$, for each $A\in C(p),$ the $\overline{f\left[A\cap X\right]}^{\mathbb{R}}$ is a compact nonempty subset of $\mathbb{R}$ and the set $\left\{\overline{f\left[A\cap S\right]}^{\mathbb{R}}: A\in C(p)\right\}$ has the finite intersection property. Thus,  $\bigcap\left\{\overline{f\left[A\cap S\right]}^{\mathbb{R}}: A\in C(p)\right\}\neq \emptyset$. Note that if $s\in \bigcap\left\{ \overline{f[A\cap S]}^{\mathbb{R}}: A\in C(p)\right\}$ and $\epsilon>0$, then $p\in \overline{f^\leftarrow \left[ [s-\epsilon, s+\epsilon]\right]}^{S\cup \{p\}}$. This is because if $A\in C(p)$, then $(s-\epsilon, s+\epsilon)\cap f[A\cap S]\neq \emptyset$ and so $A\cap f^\leftarrow \left[(s-\epsilon, s+\epsilon)\right]\neq \emptyset.$ 

		  \vskip 20pt

		  Choose $r\in \bigcap \left\{\overline{f[A\cap S]}^\mathbb{R}: A\in C(p)\right\}$ and define $F:S\cup \{p\} \rightarrow \mathbb{R}$ as
		  $$ F|_S=f \mbox{ and } F(p)=r.$$

		  Since $f$ is continuous, $F$ is continuous at each point of $S$. We must show that $F$ is continuous at $p$. Let $\epsilon>0$ be given.
		  We claim that there exists $A_0\in C(p)$ such that $f\left[A_0\cap S\right] \subseteq (r-\epsilon, r+\epsilon).$ For if this were not the case, then $\overline{f\left[A\cap S\right]}^{\mathbb{R}}\backslash (r-\frac{3\epsilon}{4}, r+\frac{3\epsilon}{4})$ is a nonempty compact subset of $\mathbb{R}$ for each $A\in C(p).$ As $\left\{\overline{f[A\cap S]}^{\mathbb{R}}\backslash (r-\frac{3\epsilon}{4}, r+\frac{3\epsilon}{4}): A\in C(p)\right\}$ has the finite intersection property, there exists $s\in \bigcap \left\{\overline{f[A\cap S]}^{\mathbb{R}}\backslash (r-\frac{3\epsilon}{4}, r+\frac{3\epsilon}{4}): A\in C(p)\right\}$. As noted in the previous paragraph, it follows that $$p\in \overline{f^\leftarrow\left[[s-\frac{\epsilon}{4},s+\frac{\epsilon}{4}]\right]}^{S\cup \{p\}}.$$


		  \vskip 10pt

		  On the other hand, since $r\in \bigcap \left\{\overline{f[A\cap S]}^\mathbb{R}: A\in C(p)\right\}$, we have $$p\in \overline{f^\leftarrow \left[[r-\frac{\epsilon}{4}, r+\frac{\epsilon}{4}]\right] }^{S\cup \{p\}}.$$

		  \vskip 15pt


		  As  $f^\leftarrow\left[[s-\frac{\epsilon}{4},s+\frac{\epsilon}{4}]\right]$ and $f^\leftarrow \left[[r-\frac{\epsilon}{4}, r+\frac{\epsilon}{4}]\right]$ are disjoint zero-sets of $S$, this is a contradiction to our hypothesis that if $Z_1$ and $Z_2$ are disjoint zero-sets of $S$, then $\overline{Z_1}^X\cap \overline{Z_2}^X=\emptyset$. 

		  \vskip 20pt

		  Thus, there exists $A_0\in C(p)$ such that $f\left[A_0\cap S\right]\subseteq (r-\epsilon, r+\epsilon).$ Thus $F[A_0]\subseteq (r-\epsilon, r+\epsilon)$ and $F$ is continuous at $p$. As $f$ was arbitrarily chosen from $C^*(S)$, it follows that $S$ is $C^*$-embedded in $S\cup\{p\}.$ \qed








		  \vskip 40pt





		  \textbf{Definition.} Let $X,Y$ be spaces, and let $f$ be a function from $X$ to $Y$. If $f[Z]$ is closed in $Y$ for any zero-set $Z$ of $X$, then $f$ is called \textbf{z-closed}.

		  \vskip 30pt






		  \subsection{Fact.}  Let $X$ and $Y$ be Tychonoff spaces, and $\pi_X: X\times Y\rightarrow X$ be the projection map. If $\pi_X$ is z-closed, $Z$ is a zero-set in $X\times Y$, and $(x,p)\in \overline{Z}^{X\times \beta Y}$, then $(x,p)\in \overline{Z\cap \left(\{x\}\times Y\right)}^{X\times \beta Y}$.

		  \vskip 20pt

		  \textbf{Proof: }Assume that $(x,p)\notin \overline{Z\cap (\{x\}\times Y)}^{X\times \beta Y}.$ Since $X\times \beta Y$ is Tychonoff, there exists a continuous function $f: X\times \beta Y \rightarrow [0,1]$ such that $f\left[\overline{Z\cap (\{x\}\times Y)}^{X\times \beta Y}\right] \subseteq \{1\}$ and $f[U]\subseteq \{0\},$ where $U$ is some neighborhood of $(x,p).$ 

		  \vskip 15pt

		  Let $Z_f=f^{\leftarrow}(0).$ So $(x,p)\in int(Z_f)$. We have $(x,p)\in \overline{Z\cap Z_f}^{X\times \beta Y}$, and so
		  $$x\in \pi_X\left[\overline{Z\cap Z_f}^{X\times \beta Y}\right] \subseteq \overline{\pi_X\left[Z\cap Z_f\right]}^X \subseteq \pi_X\left[Z\cap Z_f\right].$$

		  \vskip 10pt

		  On the other hand, since $\overline{Z\cap(\{x\}\times Y)}^{X\times \beta Y} \cap Z_f = \emptyset, $ 
		  we have $Z\cap (\{x\}\times Y) \cap Z_f=\emptyset$. Now, as $x\in \pi_X\left[Z\cap Z_f\right],$ then $(x,y)\in Z\cap Z_f\neq \emptyset$ for some $y\in Y$, hence 
		  $Z\cap Z_f\cap (\{x\}\times Y) \neq \emptyset,$  contradiction. Thus, $(x,p)\in \overline{Z\cap (\{x\}\times Y)}^{X\times \beta Y}.$\qed








		  \vskip 40pt






		  \subsection{Fact.} Let $X$ be a Tychonoff space. If $X$ is pseudocompact, then every locally finite family of nonempty open subsets of $X$ is finite.

		  \vskip 20pt

		  \textbf{Proof.} By way of contradiction, suppose that there exists a locally finite family $\mathcal{F}=\{U_i \in \tau(X): U_i\neq \emptyset, i\in \mathbb{N}\}$ of nonempty open sets. Since each $U_i$ is nonempty, choose a point $x_i\in U_i$ for each $i\in \mathbb{N}$. Since 
		  $X$ is a Tychonoff space, there exists continuous functions $f_i:X \rightarrow [0,i]$ such that $f_i(x_i)=i$ and 
		  $f_i[X \backslash U_i] \subseteq \{0\}$ for each $i \in \mathbb{N}$.

		  \vskip 20pt

		  Define the function $$f:X\rightarrow \mathbb{R} \mbox{ as } f(x)=\Sigma_{i=1}^\infty |f_i(x)|.$$ To show that $f$ is continuous, pick $x_0\in X$ and an open set $V$ of $\mathbb{R}$ containing $f(x_0)$. 
		  We can assume that  $V=(f(x_0)-\frac{1}{m}, f(x_0)+\frac{1}{m})$ for some $m \in \mathbb{N}$.
		  Since $\mathcal{F}$ is locally finite, there exists an open set $U_0 \in \tau(X)$ containing $x_0$ such that $U_0$ meets $\mathcal{F}$ only finitely many times. 
		  So we have $\{a_i\}_{i=1}^{n} \subset \mathbb{N}$ such that $U_0\cap U_{a_i} \neq \emptyset$ for $i\leq n,$ i.e., $U_0\subseteq \bigcup \left\{U_{a_i}: i\leq n\right\}.$

		  \vskip 20pt
		  As $f|_{U_0} = \Sigma_{i=1}^{n} f_{a_i} |_{U_0}$, $f$ is continuous on $U_0$. By the pasting theorem, $f$ is continuous on $X$.
		  However, since $f(x_i) \geq i$ for all $i\in \mathbb{N},$ $f$ is  not bounded. This contradicts the pseudocompactness of $X$.\qed










		  \vskip 40pt













		  \subsection{Lemma.} Let $X, Y$ be Tychonoff spaces. If $X\times Y$ is pseudocompact, then the projection map $\pi_X: X\times Y\rightarrow X$, is z-closed. 

		  \vskip 20pt

		  \textbf{Proof.} Let $Z$ be a zero-set in $X\times Y$. Suppose that $\pi_X[Z]$ is not closed in $X$. Let $p\in \overline{\pi_X[Z]}^X\backslash \pi_X[Z].$ 

		  \vskip 15pt

		  Since $Z$ is a zero-set in $X\times Y$, $Z=f^\leftarrow (0)$ for some $f\in C^*(X\times Y).$ Define $h: X\times Y\rightarrow \mathbb{R}$ by $h(x,y)=\frac{f(x,y)}{f(p,y)}$. So, $h\left[\{p\}\times Y\right] \subseteq \{1\}$ and $Z=h^\leftarrow (0).$ Without loss of generality, we can assume that the range of $h$ is $[0,1]$. 
		  \vskip 10pt

		  We will show that there are open sets $U_n, V_n$ in $X$, and $W_n$ in $Y$ for $n<\omega$ such that for $m<\omega$, the following hold: 

		  \begin{enumerate}
		  \item $p\in U_m$
		  \item $(V_m\times W_m)\cap Z \neq \emptyset$
		  \item $h\left[V_m\times W_m\right] \subseteq [0,\frac{1}{3})$
		  \item $h\left[U_m\times W_m\right] \subseteq (\frac{2}{3},1]$
			  \item $U_{m+1}\cup V_{m+1}\subseteq U_m$
			  \end{enumerate}

			  \vskip 5pt

			  First, pick $(x_1,y_1)\in Z$ and open sets $U_1,V_1\in \tau(X)$ and $W_1\in \tau(Y)$ such that $p\in U_1, x_1\in V_1, y_1\in W_1,$ and $h\left[V_1\times W_1\right] \subseteq [0,\frac{1}{3})$ and $h\left[U_1\times W_1\right] \subseteq (\frac{2}{3}, 1]$. This can be done because $h$ is continuous, $h(x_1,y_1)=0$, and $h(p,y_1)=1$.

			  \vskip 10pt

			  Now, $U_1\cap \pi_X[Z]\neq \emptyset$ because $x_1\in U_1\in \tau(X), $ and $x_1\in \overline{\pi_X[Z]}^X.$ So there is some $(x_2,y_2)\in Z$ such that $x_2\in U_1$. Find open neighborhoods $U_2$ of $p$, $V_2$ of $x_2$, and $W_2$ of $y_2$ such that $h\left[V_2\times W_2\right] \subseteq [0,\frac{1}{3}), h\left[U_2\times W_2\right]\subseteq (\frac{2}{3},1],$ and $U_2\cup V_2\subseteq U_1$. Continue by induction. 


			  \vskip 20pt
			  The family $D=\left\{V_n\times W_n: n<\omega\right\}$ is pairwise disjoint because the $V_n$'s are pairwise disjoint by our construction. If $D$ is locally finite, then by $\textbf{Fact 2.5}$, $D$ is finite. But $D$ is infinite by our definition, so $D$ cannot be locally finite. Then, there exists $(q,r)\in X\times Y$ with the property that for every neighborhood $R\times T$ of $(q,r)$, $A=\left\{n\in \mathbb{N}: (V_n\times W_n)\cap (R\times T) \neq \emptyset\right\}$ is infinite. 
			  \vskip 20pt

			  On one hand, we have $(q,r)\in \overline{\bigcup\left\{V_m\times W_m: m\in \mathbb{N}\right\}}^{X\times Y}.$ Then, 
			  $$h(q,r)\in h\left[\overline{\bigcup\left\{V_m\times W_m: m\in \mathbb{N}\right\}}^{X\times Y}\right] $$
			  $$\subseteq \overline{h\left[\bigcup\left\{V_m\times W_m: m\in \mathbb{N}\right\}\right]}^{\mathbb{R}}\subseteq \overline{[0,\frac{1}{3})}^\mathbb{R}=[0,\frac{1}{3}].$$

			  \vskip 20pt

			  On the other hand, if $n$ and $n+k$ in $A$ where $n,k\in \mathbb{N}$, then $V_{n+k}\subseteq U_{n+k-1}\subseteq \cdots \subseteq U_n$ by the way we constructed $V_n$'s and $U_n$'s. Since $(R\times T)\cap (V_{n+k}\times W_{n+k})\neq \emptyset, (R\times T)\cap (U_n\times W_n) \neq \emptyset$ as well. 
			  \vskip 10pt
			  So, $(q,r)\in \overline{\bigcup\left\{U_m\times W_m: m\in \mathbb{N}\right\}}^{X\times Y}.$ Then, 
			  $$h(q,r)\in h\left[\overline{\bigcup\left\{U_m\times W_m: m\in \mathbb{N}\right\}}^{X\times Y}\right] $$
			  $$\subseteq \overline{h\left[\bigcup\left\{U_m\times W_m: m\in \mathbb{N}\right\}\right]}^{\mathbb{R}}\subseteq \overline{(\frac{2}{3},1]}^\mathbb{R}=[\frac{2}{3}, 1].$$


				  \vskip 15pt

				  This is a contradiction, so $\pi_X[Z]$ must be closed in $X$.\qed




				  \vskip 40pt







				  \subsection{Lemma.}  Let $X,Y$ be Tychonoff spaces. If $\pi_X$ is z-closed, then $X\times Y$ is $C^*$-embedded in $X\times \beta Y$.


			  \vskip 15pt


			  \textbf{Proof:} By \textbf{Fact 2.3}, it suffices to show that if $Z_1$ and $Z_2$ are disjoint zero-sets of $X\times Y$, then $\overline{Z_1}^{X\times \beta Y} \cap \overline{Z_2}^{X\times \beta Y}=\emptyset.$

			  \vskip 15pt

			  Assume there is some point $(x,p)\in \overline{Z_1}^{X\times \beta Y}\cap \overline{Z_2}^{X\times \beta Y},$ where $x\in X$ and $p\in \beta Y\backslash Y$. By \textbf{Fact 2.4}, $$(x,p)\in \overline{Z_1\cap (\{x\}\times Y)}^{\{x\}\times \beta Y} \cap \overline{Z_2\cap (\{x\}\times Y)}^{\{x\}\times \beta Y}.$$

			  \vskip 10pt

			  Now, $Z_1\cap (\{x\}\times Y)$ and $Z_2\cap (\{x\} \times Y)$ are disjoint zero-sets in $\{x\}\times Y$. Since $\{x\}\times Y$ is $C^*$-embedded in $\{x\}\times \beta Y$, then, by the other direction of \textbf{Fact 2.3}, $$\overline{Z_1\cap (\{x\}\times Y)}^{\{x\}\times \beta Y} \cap \overline{Z_2\cap (\{x\}\times Y)}^{\{x\}\times \beta Y}=\emptyset,$$

			  a contradiction, so $\overline{Z_1}^{X\times \beta Y} \cap \overline{Z_2}^{X\times \beta Y} =\emptyset$.  \qed





			  \vskip 40pt


			  \subsection{Lemma.} Let $Y$ be an extension of the space $X$. If $X$ is pseudocompact, so is $Y$. 
			  \vskip 20pt
			  \textbf{Proof.} Let $f\in C(Y)$. Since $X$ is pseudocompact, $f|_X$ is bounded. There is $n\in \mathbb{N}$ such that $f[X]\subseteq [-n,n].$ Now, $f[Y]=f\left[\overline{X}^Y\right]\subseteq \overline{f[X]}^\mathbb{R} \subseteq \overline{[-n,n]}^\mathbb{R}=[-n,n].$ Hence, $f$ is bounded an $Y$ is pseudocompact. \qed




			  \vskip 40pt

			  \subsection{Glicksberg's Theorem.} Let $X\times Y$ be Tychonoff spaces. If $X\times Y$ is pseudocompact, then $\beta(X\times Y)=\beta X\times \beta Y$. 

			  \vskip 20pt

			  \textbf{Proof. } By \textbf{Lemma 2.6}, the projection map $\pi_X: X\times Y\rightarrow X$ is z-closed. By \textbf{Lemma 2.7}, $X\times Y$ is $C^*$-embedded in $X\times \beta Y$. By \textbf{Lemma 2.8}, the extension $X\times \beta Y$ of $X\times Y$ is pseudocompact. Using \textbf{Lemma 2.6}, \textbf{Lemma 2.7} again, and by symmetry, $X\times \beta Y$ is $C^*$-embedded in $\beta X\times \beta Y$. By the transitivity of $C^*$-embedding, $X\times Y$ is $C^*$-embedded in $\beta X\times \beta Y$, i.e., $\beta (X\times Y)=\beta X\times \beta Y$.\qed
















			  \newpage














			  \begin{center}

			  \end{center}
			  \vskip 40pt
			  \begin{center}
			  \section{Topological Facts }
			  \end{center}
			  \vskip 20pt

			  In this chapter, we will prove various facts that may be of use in the proof of Buzjakova's Theorem.


			  \vskip 20pt
			  \textbf{Fact 3.1} If $X$ is pseudocompact and $Y$ is compact, then $X\times Y$ is pseudocompact. 

			  \vskip 15pt


			  \textbf{Proof.} Let $f: X\times Y \rightarrow \mathbb{R}$. As $\{x\}\times Y$ is compact, $f\left[ \{x\} \times Y\right] $ is 
			  closed and bounded in $\mathbb{R}$ for all $x \in X$. We can define $g:X \rightarrow \mathbb{R}$ as 
			  $$g(x)=\max \{f(x,y):y\in Y\}.$$

			  Fix $x_0 \in X$, we will show that $g$ is continuous at $x_0$. Let $\epsilon >0$. 

			  \vskip 10pt


			  By our definition of $g$ and since $Y$ is compact, there exists some $y_0\in Y$ such that $g(x_0)=f(x_0,y_0)$. Let $r=f(x_0,y_0)$. Now, define the sets $U_y$'s and $V_y$'s as follows:  \vskip 15pt

			  \emph{For each $y\in Y:$}
			  \vskip 5pt

			  If $f(x_0,y) \in (r-\epsilon, r+\epsilon)$, we can find $V_{y} \in \tau(Y) $ and $U_{y} \in \tau(X)$ such that $x_0\in U_y, y_0\in V_{y}$ and $f\left[U_y \times V_y\right] \subseteq (r-\epsilon, r+\epsilon)$. In particular, since $f(x_0,y_0) \in (r-\epsilon, r+\epsilon)$, $x_0\in U_{y_0}\in \tau(X),$ $y_0\in V_{y_0}\in \tau(Y)$, and $f\left[U_{y_0} \times V_{y_0}\right] \subseteq (r-\epsilon, r+\epsilon)$
			  \vskip 5pt

			  If $f(x_0,y) \notin (r-\epsilon, r+\epsilon)$, then since $f(x_0,y)\leq \max\{f(x_0,y):y\in Y\}=r$, we must have $f(x_0,y)\leq r-\epsilon$. Hence, we can get $V_{y} \in \tau(Y) $ and $U_{y} \in \tau(X)$ such that  $x_0\in U_y, y_0\in V_{y}$, and $f\left[U_y \times V_y\right] \subseteq (-\infty, r).$ \\

			  \vskip 12pt

			  The family $\{V_y: y\in Y\}$ as defined aboved is an open cover of $Y$. By compactness, there exists $\{V_i: 1\leq i \leq n\} \subseteq \{V_y: y\in Y\}$ such that $\bigcup \{V_i: 1\leq i \leq n\} = Y$. Corresponding to $\{V_i: 1\leq i \leq n\}$, we have the set $\{U_i: 1\leq i \leq n\}$. Let $U=\bigcap \{U_i: 1\leq i \leq n\} \cap U_{y_0}$, now $U$ is an open set containing $x_0$. \\

			  \vskip 5pt

			  For any $x\in U$, we have $\max\{f(x,y):y\in Y\}<r+\epsilon$ because:
			  $$\left\{f(x,y):y\in Y\right\} = \left\{f(x,y):y\in \bigcup \{V_i: 1\leq i \leq n\}\right\}$$
			  $$=\bigcup \left\{f\left[\{x\}\times V_i\right]: 1\leq i\leq n\right\}$$
			  $$\subseteq \bigcup \left\{f\left[U_i\times V_i\right]: 1\leq i\leq n\right\}$$
			  $$ \subseteq (-\infty, r+\epsilon).$$

			  \vskip 20pt

			  One the other hand, we have $\max\{f(x,y):y\in Y\}>r-\epsilon$ because:
			  $$\max\{f(x,y):y\in Y\}\geq f(x,y_0), \mbox{and}$$
			  $$f(x,y_0)\in f\left[U\times \{y_0\}\right] \subseteq f\left[U_{y_0} \times \{y_0\}\right]
			  \subseteq f\left[U_{y_0} \times V_{y_0}\right] \subseteq (r-\epsilon, r+\epsilon).$$


			  \vskip 10 pt
			  We have now $r-\epsilon<\max\{f(x,y):y\in Y\}<r+\epsilon$ for all $x\in U$. Hence, $g[U] \subseteq (r-\epsilon, r+\epsilon)$, so $g$ is continuous on $X$. 
			  As $X$ is pseudocompact, $g$ must be bounded. Therefore, $f$ must be bounded as well. 
			  Thus, $X\times Y$ is pseudocompact. \qed


			  \vskip 40pt 





			  \textbf{Definition. } A space $X$ is \emph{countably compact} iff each countable open cover of $X$ has a finite subcover.


			  \vskip 30pt

			  \textbf{Fact 3.2} For every Hausdorff spaces $X$, the following statements are equivalent: \vskip 5pt
			  \begin{enumerate}
			  \item  The space $X$ is countably compact.
			  \item For every decreasing sequence $F_1\supseteq F_2 \supseteq \cdots $ of nonempty closed subsets of $X$, the intersection $\bigcap_{i=1}^{\infty} F_i$ is nonempty. 
			  \item Every countably infinite subset of $X$ has an accumulation point. 
			  \end{enumerate}
			  \vskip 15pt
			  \textbf{Proof.} \vskip 10pt

			  \textbf{1$\Rightarrow $2:} Let $F_1\supseteq F_2 \supseteq \cdots $ be nonempty closed subsets of $X$. If $\bigcap_{i=1}^{\infty} F_i =\emptyset$, then $\left\{X\backslash F_i : i\in \mathbb{N} \right\} $ would be an countable open cover of $X$, so there is a finite subcover $\{X\backslash F'_i : 1\leq i \leq n\} \subseteq \left\{X\backslash F_i : i\in \mathbb{N}\right\}$ such that $\bigcup \{X\backslash F'_i : 1\leq i \leq n\}= X$. Now, because the $F_i$'s are decreasing, without loss of generality, $F'_1 \supseteq F'_2 \supseteq \cdots \supseteq F'_n$. So, $X=\bigcup \{X\backslash F'_i : 1\leq i \leq n\}= X\backslash F'_n$ and $F'_n=\emptyset$, a contradiction.


			  \vskip 10pt

			  \textbf{2$\Rightarrow $1:} By way of contradiction, suppose that $X$ is not countably compact.
			  Let $\{U_i\in \tau(X): i\in \mathbb{N}\}$ be a countable cover of $X$ without an finite subcover. For each $n\in \mathbb{N}$, define $F_n=X\backslash \bigcup\{U_i: 1\leq i\leq n\}.$ For each $n$, $F_n$ is nonempty because if it is, then $\{U_i: i\in \mathbb{N}\}$ would be a finite subcover of $X$, a contradiction. Thus, we have $F_1\supseteq F_2 \supseteq \cdots $ and each $F_n$ is a nonempty closed subset of $X$. 
			  \vskip 10pt
			  Now, by our assumption, the intersection $\bigcap_{i \in \mathbb{N}} F_i$ is nonempty. So there exists some $x\in \bigcap_{i\in \mathbb{N}} F_i$. So $x\in F_i$ for all $i\in \mathbb{N}$. That means $x\notin U_i$ for all $i\in \mathbb{N}$, contradicting that $\{U_i: i\in \mathbb{N}\}$ is a cover of $X$. 

			  \vskip 15pt




			  \textbf{1$\Rightarrow $3:} By way of contradiction, suppose we have a countably infinite subset $A=\{x_i\in X: i\in \mathbb{N}\}$ with no accumulation point in $X$. Then $A$ is closed in $X$ and every point in  $A$ is an isolated point with respect to $A$. For each $x_i\in A$, there is $x_i \in U_{x_i}\in \tau(X)$ such that $U_{x_i} \cap A = \{x_i\}$. So $\{X\backslash A\} \cup \{U_{x_i} \in \tau(X): i\in \mathbb{N}\}$ is an countable open cover of $X$ that yields no finite subcover, contradicting that $X$ is countably compact. 

			  \vskip 15pt


			  \textbf{3$\Rightarrow$1:} By way of contradiction, suppose that  $\{U_i\in \tau(X): i\in \mathbb{N}\}$ is a countable cover of $X$ with no finite subcover. Then, by the equivalence of \textbf{1} and \textbf{2}, there exists a decreasing sequence $F_1\supset F_2\cdots $ of nonempty closed subsets of $X$ such that $\bigcap_{i\in \mathbb{N}} F_i =\emptyset$. We define the set $A=\{x_n: n\in \mathbb{N}\}$ by $x_i\in F_i$ for each $i\in \mathbb{N}$. If $A$ is finite, there must be some $x_j\in A$ such that $x_j$ belongs to infinitely many $F_i$'s, and since $F_i$'s are decreasing, $x_j$ would have to be in all $F_i$'s. Contradicting $\bigcap_{i\in \mathbb{N}} F_i=\emptyset$. 
			  Hence, $A$ is an infinite set. By our assumption, $A$ has an accumulation point. Let $x$ be an accumulation point of $A$.

			  \vskip 10pt
			  Since $\bigcap_{i\in \mathbb{N}} F_i =\emptyset$, there exists an $i$ such that $x\notin F_i$. Now, $U=X\backslash F_i$ is an open set that contains $x$, and $U$ does not contain any point of the set $\{x_j: j\geq i\}\subseteq F_i$.  That is, $U\cap A \subseteq \{x_1,\dots, x_i\}$ and $x$ is not an accumulation point of $A$, a contradiction.   \qed






			  \vskip 40pt

			  \textbf{Definition. } For an ordinal $\tau$, let $T(\tau)$ denote the space of ordinals less than $\tau$. 

			  \vskip 25pt


			  \textbf{Fact 3.3} Let $\tau$ be uncountable regular cardinal. Let $g: T(\tau)\rightarrow \mathbb{R}$ be continuous.
			  Then  $g$ is constant on $[\kappa, \tau)$ for some $\kappa<\tau$.
			  \vskip 15pt
			  \textbf{Proof.}
			  Let $\alpha<\tau$. If $A=\{\alpha_n: n\in \mathbb{N}\}$ is a countably infinite subset of $[\alpha,\tau)$, then since $\tau$ is uncountable regular, $\beta=\sup \left\{\alpha_n: n\in \mathbb{N}\right\}<\tau$ and $A\subseteq [\alpha, \beta],$ a compact subset.
			  So $A$ has an accumulation point. Thus $[\alpha,\tau)$ must be countably compact by \textbf{Fact 3.2}.

			  \vskip 15pt

			  Since $g$ is continuous, $g\left[[\alpha, \tau)\right]$ is a countably compact subset of $\mathbb{R}$. In metric spaces, countably compact is equivalent to compact because metric spaces are Lindel$\ddot{o}$f. Hence, $g\left[[\alpha,\tau)\right]$ is compact for all $\alpha<\tau$. 
			  Thus, there exists $p \in \bigcap_{\alpha<\tau} g\left[[\alpha, \tau)\right]$. To show that $p$ is unique, suppose that there exists $q \in \bigcap_{\alpha<\tau} g\left[[\alpha, \tau)\right]$. 

			  \vskip 15 pt

			  There exists some $\alpha_0 \in [0,\tau)$ such that $g(\alpha_0)=p$. As $q\in g\left[[\alpha_0+1,\tau)\right]$, there exists $\alpha_1\in [\alpha_0+1,\tau)$ such that $g(\alpha_1)=q.$ As $p\in g\left[[\alpha_1+1,\tau)\right]$, there exists $\alpha_2\in [\alpha_1+1,\tau)$ such that $g(\alpha_2)=p.$ We continue this process by induction. We have now: 
			  $$p=g(\alpha_0)=g(\alpha_2)=g(\alpha_4)=\cdots$$
			  $$q=g(\alpha_1)=g(\alpha_3)=g(\alpha_5)=\cdots$$

			  Let $\beta=\sup\{\alpha_n: n<\omega\}$, which exists because $cf(\tau)>\omega$. By the continuity of $g$, $g(\beta)=\lim_{n<\omega} g(\alpha_n)$. Thus, 
			  $$p=\lim_{n<\omega} g(\alpha_{2n})=g(\beta)=\lim_{n<\omega} g(\alpha_{2n+1})=q$$

			  \vskip 10pt

			  So, $\bigcap_{\alpha<\tau} g\left[[\alpha, \tau)\right]=\{p\}$. Since $\mathbb{R}$ is locally compact, for each $n<\omega$, we can find some $\gamma_n\in T(\tau)$ such that $g\left[[\gamma_n,\tau)\right] \subseteq (p-\frac{1}{n}, p+\frac{1}{n}).$ Let $\kappa=\sup_{n<\omega} \gamma_n$. So, we have 
			  $$g\left[[\kappa, \tau)\right] \subseteq \bigcap_{n<\omega} (p-\frac{1}{n}, p+\frac{1}{n})=\{p\}.$$\qed


			  \vskip 40pt

			  \textbf{Fact 3.4} Let $\tau$ be an uncountable regular cardinal. The space $T(\tau)$ is pseudocompact.
			  \vskip 15 pt
			  \textbf{Proof.} Let $g: T(\tau) \rightarrow \mathbb{R}$ be a continuous function. 
			  By \textbf{Fact 3.3}, there exists $\kappa <\tau$ such that $g \left[ [\kappa, \tau) \right]=\{r\}$ for some $r\in \mathbb{R}$. As $[0,\kappa +1]$ is compact, $g$ is bounded on $[0,\kappa +1]$. Thus, $g[T(\tau)]=g[[0, \kappa +1]]\cup \{r\}$ is bounded. \qed


			  \vskip 40pt

			  \textbf{Fact 3.5} Let $X$ be a pseudocompact Tychonoff space and $\tau=|\beta X|^+$. Then,  $X\times T(\tau)$ is pseudocompact.
			  \vskip 20pt


			  \textbf{Proof.} Let $f: X\times T(\tau)\rightarrow \mathbb{R}$ be continuous. By \textbf{Fact 3.4} , the ordinal space $T(\tau)$ is pseudocompact.
			  \vskip 10pt 

			  By \textbf{Fact 3.3}, for each $x\in X$, there exists $\kappa_x<\tau$ such that $f$ is constant on $\{x\}\times [\kappa_x, \tau)$. 
			  As $cf(\tau)>|X|,$  $\kappa=\sup_{x\in X} \{\kappa_x: x\in X\}<\tau.$ 
			  Now, $f\left[X\times [0,\kappa+1]\right]$ is bounded because $X\times [0,\kappa+1]$ is pseudocompact by \textbf{Fact 3.1}.
			  For $\alpha\geq \kappa, f(x,\alpha)=f(x,\beta).$ Thus, $f\left[X\times [\kappa,\tau)\right]=f\left[X\times\{\kappa\}\right]$ which is bounded because $X$ is pseudocompact. The boundedness of $f\left[X\times [0,\kappa+1]\right]$ and $f\left[X\times [\kappa,\tau)\right]$ gives us that $f\left[X\times T(\tau)\right]$ is bounded. Hence, $X\times T(\tau)$ is pseudocompact.\qed






			  \vskip 40pt



			  \textbf{Fact 3.6} Let $\tau$ be an uncountable regular cardinal. Let $T(\tau)$ be the space 
			  of all ordinal numbers less than $\tau$. Let $A_\alpha$ be a closed, unbounded subset of $T(\tau)$. Let $\gamma\in T(\tau).$ Then, $\bigcap \{A_\alpha : \alpha<\gamma\}$ is closed, unbounded and $\left| \bigcap \{A_\alpha: \alpha<\gamma\} \right|=\tau$.



			  \vskip 15pt
			  \textbf{Proof.} 

			  We will construct the set $\{p_\alpha: \alpha<\tau\}$ by transfinite induction.

			  \vskip 10pt
			  \texttt{\textbf{Step 1.}}

			  Pick any element $a_{1,1}\in A_1$, we can find some element $a_{1,2}\in A_2$ such that $a_{1,2}>a_{1,1}$ because $A_2$ is unbounded. Then, 
			  by continuing this process, we can define $a_{1,n}$ in the same way, for all $n<\omega$. 
			  For all $\alpha<\gamma$,
			  If $\alpha$ is a successor ordinal, then since $A_\alpha$ is unbounded, we can find some $a_{1,\alpha} \in A_\alpha$ such that $a_{1,\alpha}>a_{1,\alpha-1}$. If $\alpha$ is a limit ordinal, then let $\beta=\sup_{\kappa<\alpha} \{a_{1,\kappa}\}$, which exists because $\alpha<cf(\tau)$. Now, since $A_\alpha$ is unbounded, we can find some $a_{1,\alpha}\in A_\alpha$ such that $a_{1,\alpha} >  \beta$.

			  Thus, we have defined the set $\{a_{1,\alpha}: \alpha <\gamma\}$. Let $\beta_1=\sup\{a_{1,\alpha}: \alpha <\gamma\}$, which exists because $\gamma<cf(\tau)$. 

			  \vskip 10pt

			  \texttt{Step N.}
			  Let $a_{n,1}\in A_1$ be such that $a_{n,1} > \beta_{n-1}$. Let $a_{n,2}\in A_2$ be such that $a_{n,2} > a_{n,1}$. Now continuing the same way as in Step 1, we can define $a_{n,\alpha}$ for all $\alpha<\gamma$. Let $\beta_n=\sup\{a_{n,\alpha}: \alpha <\gamma\}$. 
			  \vskip 10pt
			  So, we have contructed the set $\left\{a_{n,\alpha}: n<\omega, \alpha<\gamma\right\}$.

			  \vskip 10pt

			  For all $\alpha<\gamma, \lim_{n<\omega} a_{n,\alpha} \in A_\alpha$ because $A_\alpha$ is closed. Moreover, if $\alpha,\alpha' <\gamma$, then $\lim_{n<\omega} a_{n,\alpha}= \lim_{n<\omega} a_{n,\alpha'}$. So if we define $p_1=\lim_{n<\omega} a_{n,\alpha}$ for some $\alpha<\gamma$, then $p_1\in \bigcap \{A_\alpha: \alpha<\gamma\}$. 
			  \vskip 10pt


			  For all $\alpha <\tau$, if $\alpha$ is an isolated ordinal, then we start from $p_{\alpha-1}\in A_1$ in Step 1 again, and define $p_\alpha$ the same way as we did for $p_1$. 
			  If $\alpha$ is a limit ordinal, then we let $p_\alpha= \sup \{p_\kappa: \kappa<\alpha\}$. This exists because $\alpha<cf(\tau)$. 

			  \vskip 10pt

			  We've finished contruction of the set $\{p_\alpha: \alpha<\tau\}\subseteq T(\tau)$.  From the way we contructed it, this set is closed, unbounded and its cardinality is $\tau$.\qed


			  \vskip 30pt

			  \textbf{Fact 3.7} Let $X$ be a Tychonoff space and $|X|>\aleph_0$. Let $\tau=|\beta X|^+$. Then, $T(\tau)$ can be condensed onto $T(\tau+1).$ Moreover, for any space $X, X\times T(\tau)$ condenses onto $X\times T(\tau+1).$

			  \vskip 15pt

			  \textbf{Proof.} Define $g:T(\tau) \rightarrow T(\tau+1)$ by $g(0)=\tau$, $g(\alpha)=\alpha-1$ for all $0<\alpha<\omega$, and $g(\alpha)=\alpha$ for $\omega\leq \alpha <\tau$. Now, $g$ is one-to-one and onto. Note that $g$ is continuous at $\omega$ because if $(\beta, \omega]$ is an open set containing $g(\omega)$,then $(\beta+1,\omega]$ is an open set such that $g\left[(\beta+1, \omega]\right] \subseteq (\beta, \omega],$ and $g$ is continuous on all $\alpha<\omega$ because $\{\alpha\} \in T(\tau)$; finally, $g$ is continuous on all $\alpha>\omega$ because $g|_{(\omega, \tau)}$ is the identity function. Thus, $T(\tau)$ can be condensed onto $T(\tau+1).$

					      \vskip 10pt


					      Now, define $h: X\times T(\tau)\rightarrow X\times T(\tau+1)$ by $h(x,\alpha)=(x,g(\alpha)).$ Since $g$ is one-to-one, onto, and continuous, then, $h$ must also be one-to-one, onto, and continuous. \qed


					      \vskip 40pt


					      \textbf{Fact 3.8} Let $Z$ be a Tychonoff space. Let $A$ be a closed subset of $Z$,  and $B$ be a compact subset of $\beta Z$ such that $A$ and $B$ are disjoint. Moreover, the set $A\cup B$ is not compact in $\beta Z$. 
					      Then, there exists a system $D=\{D_\alpha\}_{\alpha<l}$ satisfying the following conditions:
					      \begin{enumerate}
					      \item For each $  \alpha  $, the set $ D_\alpha  $ is nonempty and closed in $A$.
					      \item For $ \alpha>\beta, D_\alpha \subseteq D_\beta  $ and if $  \beta $ is a limit ordinal number, then\\$  D_\beta=\bigcap \{D_\alpha:\alpha <\beta\}.$
					      \item $ \bigcap\{D_\alpha: \alpha<l\}=\emptyset . $
					      \item $\overline{D_1}^{\beta Z}\cap B =\emptyset.$

					      \end{enumerate}

			  \textbf{Proof.} \\
			      Since $A\cup B $ is not compact, there is an open cover $\mathcal{C} \subset \tau(\beta Z)$ of $A\cup B$ that has no finite subcover.  Since $B$ is compact and $\mathcal{C}$ covers $B$, there is a finite subcover $\{C_i:1\leq i \leq n\}\subseteq \mathcal{C}$ such that $B\subseteq  \bigcup\left\{C_i:1\leq i \leq n\right\}$.
			      \vskip 10pt

			      Let $E=A\backslash \bigcup \{C_i:1\leq i \leq n\}$. $E$ is closed in $A$. $E$ is nonempty because otherwise $\{C_i:1\leq i \leq n\}$ covers $A$ as well as $B$, a contradiction. Furthermore, $E$ is not compact. If $E$ is compact, we can get a finite subcover $\{C'_i:1\leq i\leq n'\}$ from $\mathcal{C}$. Then, $\{C'_i:1\leq i\leq n'\} \cup \{C_i:1\leq i \leq n\}$ is a finite subcover that covers $A\cup B$, a contradiction. 


			      \vskip 10pt


			      As $E$ is not compact, we can find an open cover $\mathcal{F}\subseteq \tau(\beta Z)$ such that no finite subcover of $\mathcal{F}$ covers $E$. 
			      without loss of generality, we can assume that $|\mathcal{F}|=L(E)$, the Lindel$\ddot{o}$f number of $E$. We can well-order $\mathcal{F}$, so $\mathcal{F}=\left\{F_\alpha: \alpha<L(E)\right\}$.
			      Define $D_\alpha=E\backslash \bigcup \left\{F_\gamma: \gamma<\alpha\right\}$ for each $\alpha < L(E)$. We shall verify that $D$ satisfies all four condictions:


			      \begin{enumerate}
			      \item For each $  \alpha  $, the set $ D_\alpha  $ is nonempty and closed in $A$.\\
				  \texttt{Proof-} Each $D_\alpha$ is nonempty because if $D_\alpha=\emptyset$ for some $\alpha$, then $E\backslash \bigcup \left\{F_\gamma: \gamma<\alpha\right\}=\empty$ and so $E \subseteq \{F_\gamma: \gamma<\alpha\}$. However, since $\alpha<L(E)$, we have a contradiction. So $D_\alpha$ is nonempty. Moreover, $D_\alpha$ is closed in $A$ because it is closed in $E$, and $E$ is closed in $A$.
				  \vskip 10pt
				  \item For $ \alpha>\beta, D_\alpha \subseteq D_\beta  $ and if $  \beta <L(E) $ is a limit ordinal number, then\\$  D_\beta=\bigcap \{D_\alpha:\alpha <\beta\}.$\\
																			      \texttt{Proof-} By the way we defined the ${D_\alpha}'s$, $D_\alpha\subseteq D_\beta$ if $\alpha>\beta$. If $\beta$ a limit ordinal number, and 
																			      if $D_\beta \neq \bigcap\left\{D_\alpha:\alpha<\beta\right\}, $ then we replace $D_\beta$ with the set $\bigcap\left\{D_\alpha:\alpha<\beta\right\}$, which is nonempty and\\ closed in $A$. So now, $ D_\beta=\bigcap \{D_\alpha:\alpha <\beta\}.$
																			      \vskip 10pt
																			      \item $ \bigcap\{D_\alpha: \alpha< L(E)\}=\emptyset . $\\
																				  \texttt{Proof-} This is true because $ \bigcap\{D_\alpha\}= E\backslash\bigcup\left\{F_\gamma: \gamma<L(E)\right\}=\emptyset.$

																				  \vskip 10pt
																				  \item $\overline{E}^{\beta Z} \cap B =\emptyset.$\\
																				      \texttt{Since $\overline{E}^{\beta Z} \cap B=\emptyset$, and $D_1\subseteq E$, then $\overline{D_1}^{\beta Z}\cap B=\emptyset.$}	 
																				      \end{enumerate} \qed



																				      \vskip 30pt



																				      \textbf{Fact 3.9} Let $X$ be a Tychonoff space. If $B_1$, $B_2$ are subsets of $X$ such that $\overline{B_1}^{\beta X} \cap \overline{B_2}^{\beta X} \neq \emptyset,$ then $B_1$ and $B_2$ are not completely separated in $X$. 


																				      \vskip 15pt


																				      \textbf{Proof.} Suppose $B_1$ and $B_2$ are completely separated in $X$. Then there exists a continuous function function $f:X\rightarrow [0,1]$ such that 
																				      $f\left[B_1\right] \subseteq\{0\}$ and $f\left[B_2\right]\subseteq \{1\}.$ Let $\bar{f}: \beta X\rightarrow [0,1]$ be the extension of $f$. The sets $\bar{f}^\leftarrow(0)$ and $\bar{f}^{\leftarrow}(1)$ are closed in $\beta X$ such that $\overline{B_1}^{\beta X} \subseteq \bar{f}^\leftarrow(0)$ and $\overline{B_2}^{\beta X} \subseteq \bar{f}^\leftarrow(1).$ Since $\bar{f}^\leftarrow(0)\cap\bar{f}^\leftarrow(1)=\emptyset,$ we have $\overline{B_1}^{\beta X}\cap \overline{B_2}^{\beta X}=\emptyset,$ a contradiction.\qed 


																				      \vskip 30pt


																				      \textbf{Fact 3.10} Let $X$ be a Tychonoff space. If $X$ is locally compact, then $X$ can be condensed onto a compact space. 

																				      \vskip 15pt


																				      \textbf{Proof.} Let $X\cup \{\infty\}$ be the one-point compactification of $X$. Pick any $x_0\in X$. Let $K$ be the space $X\cup \{\infty\}$ with the point $\infty$ identified with $x_0$. In $K$, the open sets containing $x_0$ are of the form $U_{x_0} \cup V_{\infty},$ where $U_{x_0}$ is any open set containing $x_0$ in $X$, and $V_\infty$ is any open set containing $\infty$ in $X\cup \{\infty\}.$ For $x\in K\backslash \{x_0\},$ the open sets containing $x$ in $K$ are same as the open sets containing $x$ in $X$. 
																				      \vskip 10pt

																				      $K$ is compact with the topology we've just defined. Let $f:X\times K$ be the identity map. So, $f$ is one-to-one and onto. Let $U_{x_0}\cup V_\infty$ be an open set containing $f(x_0)=x_0,$ then $U_{x_0}$ is an open set in $X$ such that
																				      $f\left[U_{x_0}\right]=U_{x_0}\subset U_{x_0}\cup V_\infty.$  So $f$ is continuous on $x_0$, as well as on other points of $X$. Hence, $X$ can be condensed onto $K$.\qed


																				      \vskip 40pt


																				      \newpage
																				      \begin{center}
																				      \textbf{CHAPTER IV}
																				      \end{center}
																				      \vskip 40pt
																				      \begin{center}
																				      \textbf{Buzjakova's Theorem}
																				      \end{center}
																				      \vskip 20pt
																				      In this chapter, we will prove the Buzjakova's Theorem. Before proving the theorem, we will need to make a remark and proving three lemmas. The last two lemmas, \textbf{Lemma 4.3} and \textbf{Lemma 4.4}, are essential in the proof of Buzjakova's Theorem.
																				      \vskip 20pt


																				      The following remark is a corollary of Glicksberg's Theorem, and it will be used throughout the proof of Buzjakova's Theorem without being explicitly referenced.

																				      \vskip 20pt






																				      \textbf{Remark 4.1} Let $X$ be a pseudocompact space and $|X|\geq \aleph_0.$ Then 


																				      \begin{enumerate}
																				      \item $\beta(X\times T(|\beta X|^+ +1))=\beta X\times T(|\beta X|^+ +1)$
																				      \item $\beta(X\times T(|\beta X|^+))=\beta X\times T(|\beta X|^+ +1)$

																				      \end{enumerate}


																				      \vskip 20pt


																				      \textbf{Proof.} Since $X$ is pseudocompact and  $T(|\beta X|^+ +1)$ is compact, by \textbf{Fact 3.1}, $X\times T(|\beta X|^+ +1)$ is pseudocompact. By Glicksberg's Theorem, $\beta(X\times T(|\beta X|^+ +1))=\beta X\times \beta T(|\beta X|^+ +1)=\beta X\times T(|\beta X|^+ +1).$ This proves the first statement.
																				      \vskip 15pt

																				      By \textbf{Fact 3.5}, $X\times T(|\beta X|^+)$ is pseudocompact. By Glicksberg's Theorem, $\beta(X\times T(|\beta X|^+))=\beta X\times \beta T(|\beta X|^+)=\beta X\times T(|\beta X|^+ +1)$. This proves the second statement. \qed

																				      \vskip 40pt


																				      \textbf{Lemma 4.2}\\
																					  Let $X,Y$ be Tychonoff spaces and $cX, cY$ be $T_2$ compactifications of $X,Y$.
																					  Let $f$ be a continuous function from $X$ onto $Y$, $\bar{f}: cX\rightarrow cY$ be the continuous extension of $f$. Let $A$ be a closed subset of $X$. Then we have: \\
																					      \emph{(i) If $f[A]$ is not closed in $Y$ then there exists an element $x\in \overline{A}^{cX} \backslash A$ such that $\bar{f}(x) \in \overline{f[A]}^{Y}\backslash f[A].$}\\
																					      \emph{(ii) If $f[A]$ is closed in $Y$ then for any element $x \in \overline{A}^{cX} $ for which $\bar{f}(x)\in Y $ holds, we have $\bar{f}(x) \in f[A]$ holds.}\\

																					      \vskip 20pt
																					      Before we prove \emph{(i)} and \emph{(ii)}, we note that $\overline{f[A]}^{cY} = \bar{f}\left[\overline{A}^{cX}\right]$ as $\bar{f}$ is a closed function and $\bar{f}\left[\overline{A}^{cX}\right]=\overline{\bar{f}[A]}^{cY}=\overline{f[A]}^{cY}$. 

																					      \vskip 20pt

																					      \textbf{Proof of (i): }\\ 
																					      By the above,  $\overline{f[A]}^{cY} = \bar{f}\left[\overline{A}^{cX}\right]$. So, $\overline{f[A]}^Y\backslash f[A]=\overline{f[A]}^{cY}\cap Y \backslash f[A] = \bar{f}\left[\overline{A}^{cX}\right] \cap Y \backslash f[A].$ \vskip 10pt
																					      As $f[A]$ is not closed in $Y$, we have $\bar{f}\left[\overline{A}^{cX}\right]\cap Y\backslash f[A]\neq \emptyset$. So there exists $x\in \overline{A}^{cX}$ such that $\bar{f}(x)\in \bar{f}\left[\overline{A}^{cX}\right]\cap Y\backslash f[A]=\overline{f[A]}^{cY}\cap Y \backslash f[A]=\overline{f[A]}^Y\backslash f[A],$ as required.



																					      \vskip 25pt

																					      \textbf{Proof of (ii): }\\By the above, we have $\bar{f}\left[\overline{A}^{cX}\right] = \overline{f[A]}^{cY}$. So,  $\bar{f}\left[\overline{A}^{cX}\right]\cap Y = \overline{f[A]}^{cY} \cap Y$. As $f[A]$ is closed in $Y$, so  $\overline{f[A]}^{cY} \cap Y=\overline{f[A]}^Y=f[A]$. 
																					      So now we have $\bar{f}\left[\overline{A}^{cX}\right]\cap Y= f[A]$, and it gives us that if $x \in  \overline{A}^{cX} $ and if $\bar{f}(x) \in Y$, then $\bar{f}(x) \in f[A]$, as required. \qed




																					      \vskip 40pt



																					      \textbf{Lemma 4.3} Let $X$ be a pseudocompact space and $f$ be a continuous mapping of $X\times T(|\beta X|^+) $
																					      onto a Tychonoff space $Z$. By \textbf{Remark 4.1}, there is a continuous function $\bar{f}: \beta X \times T(|\beta X|^+ +1) \rightarrow \beta Z$ that extends $f$. Let $C_1=\{x\in  \beta X \backslash X: |\bar{f}[\{x\} \times T(|\beta X|^+)] \cap Z|=|\beta X|^+\}$.
																					      If $x\in C_1$, there exists an element $y_x \in X$ and a subset $T_x \subseteq T(|\beta X|^+)$ satisfying the following conditions: 
																					      \begin{enumerate}
																					      \item $|T_x|=|\beta X|^+$
																					      \item The set $T_x$ is closed in $T(|\beta X|^+).$
																					      \item For any ordinal number $\alpha \in T_x, \bar{f} (x,\alpha)=f(y_x,\alpha)$ holds.
																					      \end{enumerate}

																					      \vskip 20pt

																					      \textbf{Proof:} \\
																						  Let $x\in C_1$. Since $f$ is onto, we can write $Z=\bigcup \{f[\{y\}\times T(|\beta X|^+)]: y\in X\}$. Then,  $$\bar{f}[\{x\}\times T(|\beta X|^+)] \cap Z$$
																						  $$=\bar{f}\left[\{x\}\times T(|\beta X|^+)\right] \cap \left(\bigcup \left\{f[\{y\}\times T(|\beta X|^+)]: y\in X\right\}\right)$$
																						  $$=\bigcup \left\{\bar{f}\left[\{x\}\times T(|\beta X|^+)\right] \cap f\left[\{y\}\times T(|\beta X|^+)\right]: y\in X \right\}.$$

																						  \vskip 5pt

																						  As $|\beta X|^+$ is regular, $|X|<|\beta X|^+=cf(|\beta X|^+).$ So, at lease one of the terms of our union has cardinality $|\beta X|^+$. \\
																												Let that term be:
																												\begin{center}
																												$ \bar{f}\left[\{x\}\times T(|\beta X|^+)\right] \cap f\left[\{y_x\} \times T(|\beta X |^+)\right]$.
																												\end{center}
																												Fix this $y_x$. We will now construct the set $T_x=\{\beta_\alpha: \alpha < |\beta X|^+\}$ by transfinite induction:\\

																												\texttt{Step 1.}\\ Pick two ordinals, $\alpha_1$ and $ \alpha^1$ from $T(|\beta X|^+)$ such that $\bar{f}(x,\alpha_1)=f(y_x, \alpha^1)$. This can be done because $\bar{f}[\{x\}\times T(|\beta X|^+)] \cap f[\{y_x\} \times T(|\beta X |^+)]$ is nonempty.\\

																												\texttt{Induction Hypothesis.}\\ For each $k<n$, we can find two ordinals $\alpha_k$ and $\alpha^k$ from $T(|\beta X|^+)$,  satisfying the following conditions: \\
																												    \texttt{a)} $\bar{f}(x,\alpha_k)=f(y_x,\alpha^k)$.	\\	
																												    \texttt{	b)} $\alpha_k> \max\{\alpha_{k-1}, \alpha^{k-1}\}$.\\	
																												    \texttt{c)} $\alpha^k> \max\{\alpha_{k-1}, \alpha^{k-1}\}$.\\


																												    \texttt{Step N.}\\ Let $\alpha=\max\{\alpha_{n-1}+1,\alpha^{n-1}\}.$ Now,  \vskip 10pt    
																												    $$|\beta X|^+ $$
																												    $$= \left| \bar{f}\left[\{x\}\times T(|\beta X|^+)\right] \cap f\left[\{y_x\} \times T(|\beta X |^+)\right]\right|$$
																												    $$=\left| \bar{f}\left[\{x\}\times T(|\beta X|^+)\right] \cap f\left[\{y_x\} \times T(\alpha)\right]\right|$$

																												    $$+\left| \bar{f}\left[\{x\}\times T(\alpha)\right] \cap f\left[\{y_x\} \times T(|\beta X |^+)\right]\right|$$

																												    $$+\left| \bar{f}\left[\{x\}\times T(|\beta X|^+)\backslash T(\alpha)\right] \cap f\left[\{y_x\} \times T(|\beta X |^+)\backslash T(\alpha)\right]\right|$$\\

																												    \vskip 10pt

																												    Since the first two terms of the sum both have cardinality no more than $\alpha$, where $\alpha <|\beta X|^+$, the third term must have cardinality equal to $|\beta X|^+$. Otherwise the sum of these three terms will not add up to $|\beta X|^+$. Obviously, $\bar{f}[\{x\}\times T(|\beta X|^+)\backslash T(\alpha)] \cap f[\{y_x\} \times T(|\beta X |^+)\backslash T(\alpha)] $ is nonempty. So, we can pick two ordinals $\alpha_n$ and $\alpha^n$ from $T(|\beta X|^+)\backslash T(\alpha)$ such that these conditions hold: \\
																													\texttt{a)} $\bar{f}(x,\alpha_n)=f(y_x,\alpha^n)$.	\\	
																													\texttt{	b)} $\alpha_n> \alpha= \max\{\alpha_{n-1}, \alpha^{n-1}\}$.\\	
																													\texttt{c)} $\alpha^n> \alpha= \max\{\alpha_{n-1}, \alpha^{n-1}\}$.\\

																													This completes \texttt{Step N.} Hence, we can define $\alpha_n$ and $\alpha^n$ for all $n<\omega$.\\
																													    \vskip 10 pt
																													    Let $\beta_1=\sup\{\alpha_n:n<\omega\}=\sup\{\alpha^n: n<\omega \}.$ Such $\beta_1\in T(|\beta X|^+)$ exists because by condition \texttt{b)} and \texttt{c)}, the sup's must equal, provided they exist, and indeed, the existence follows from $cf(|\beta X|^+) > \omega$. 

																													    \vskip 15pt 

																													    Since $f$ is continuous, $\{f(y_x,\alpha^n): n<\omega\}$ converges to $f(y_x,\beta_1)$.\\
																														Since $\bar{f}(x,\alpha_n)=f(y_x,\alpha^n)$ for all $n<\omega$, $\{\bar{f}(x,\alpha_n)\: n<\omega\}$ converges to $f(y_x,\beta_1)$ also. At the same time, since $\bar{f}$ is continuous, $\left\{\bar{f}(x,\alpha_n): n<\omega\right\}$ converges to $\bar{f}(x,\beta_1).$ Hence, $\bar{f}(x, \beta_1)=f(y_x,\beta_1).$

																														\vskip 20pt

																														Now, we will define all the other $\beta_\alpha$'s by transfinite induction: \\

																														\texttt{Induction Hypothesis.}\\
																														    Let $\beta_\alpha$ be defined for all $\alpha<\gamma$ such that $\bar{f}(x,\beta_\alpha)=f(y_x,\beta_\alpha)$.\\

																														    \texttt{Step $\gamma$(isolated ordinal).}\\
																															As $\bar{f}[\{x\}\times T(|\beta X|^+)\backslash T(\beta_{\gamma-1})] \cap f[\{y_x\} \times T(|\beta X |^+)\backslash T(\beta_{\gamma-1})]$ is nonempty, we can find a pair of ordinals in  $T(|\beta X|^+)\backslash T(\beta_{\gamma-1})$, enabling us to start from \texttt{Step 1} again with those two ordinals. Then, we can constuct $\beta_\gamma$ in the way as we did for $\beta_1$.\\


																															\texttt{Step $\gamma$(limit ordinal).}\\
																															    Define $\beta_\gamma = \sup\{\beta_\alpha: \alpha<\gamma\}.$ Again, the $\sup$ exists because $\gamma < |\beta X|^+= cf|\beta X|^+.$ Furthermore, $\bar{f}(x,\beta_\gamma)=f(y_x,\beta_\gamma)$ by the continuity of $\bar{f}$.\\


																															    By defining the $\beta_\gamma$'s this way for all $\gamma <|\beta X|^+$, we have just successfully constructed the set $T_x=\{\beta_\alpha: \alpha < |\beta X|^+\}$. By the way $T_x$ and $y_x$ were defined, conditions 1 and 3 as required by the lemma automatically follow.  $T_x=\{\beta_\alpha: \alpha < |\beta X|^+\}$ is
																															    closed in $T(|\beta X|^+)$ because we defined $\beta_\gamma = \sup\{\beta_\alpha: \alpha<\gamma\}$ if $\gamma$ is a limit ordinal. Hence condition 2 of the lemma follows as well. This proves the Lemma.\qed



																															    \vskip 40pt




																															    \textbf{Lemma 4.4} Let $X$ be a pseudocompact space, and let $f$ be a continuous one-to-one function from $X\times T(|\beta X|^+)$ onto $Z$. We are given two sets: \\
																																$C_1=\{x\in \beta X\backslash X: \left|\bar{f}[\{x\}\times T(|\beta X|^+)]\cap Z\right|=|\beta X|^+\}.$\\
																																$C_2=\{x\in \beta X\backslash X: \left|\bar{f}[\{x\}\times T(|\beta X|^+)]\cap Z\right|<|\beta X|^+ $ and for any $\alpha\in T(|\beta X|^+),$ there exists $\alpha_1\in T(|\beta X|^+)$ such that $\alpha_1 >\alpha$ and $\bar{f}(x,\alpha_1) \in Z \}.$

																																Then, $C_1\cap \overline{C_2}^{\beta X}=\emptyset$ and $X\cap \overline{C_2}^{\beta X}=\emptyset$.

																																\vskip 15pt

																																\textbf{Proof.} 
																																For each $y\in C_2$, let $Z_y=\bar{f}\left[\{y\}\times T(|\beta X|^+)\right] \cap Z$. \\
																																			      Since $\{y\}\times T(|\beta X|^+) = \bigcup \left\{ \{y\} \times T(|\beta X|^+) \cap \bar{f}^{-1}(z):z\in Z_y \right\}$, we have:
																																			      \begin{center}
																																			      $\left|\beta X\right|^+ =\left| \{y\}\times T(|\beta X|^+)\right| = \left| \bigcup \left\{ \{y\} \times T(|\beta X|^+) \cap \bar{f}^{-1}(z):z\in Z_y \right\}\right|.$
																																			      \end{center}
																																			      Since $\left|Z_y\right|< cf(\left|\beta X\right|^+),$ at least one of the terms in the union must have cardinality equal to $\left|\beta X\right|^+$. So there exists $z\in Z_y$ such that 

																																			      $$\left|  \{y\} \times T(|\beta X|^+) \cap \bar{f}^{-1}(z) \right|=\left|\beta X\right|^+.$$ 

																																			      \vskip 15pt

																																			      Define $A_y=\pi_2 \left[ \{y\} \times T(|\beta X|^+) \cap \bar{f}^{-1}(z) \right]$, where $\pi_2: \beta X \times 
																																			      T(|\beta X|^+) \rightarrow T(|\beta X|^+)$ is the projection map. Thus, we defined $A_y\subseteq T(\left|\beta X\right|^+)$ such that $\left|A_y\right|=\left|\beta X\right|^+$ and $\bar{f}[\{y\}\times A_y]=\{z\}.$ Since $z\in Z$ and $f$ is onto, $z=f(x_y,\alpha_y)$ for some $(x_y,\alpha_y)\in X\times T(|\beta X|^+).$ Hence $\bar{f}\left[\{x\}\times A_y\right]=\{f(x_y,\alpha_y)\}$ for some $(x_y, \alpha_y)\in X\times T(|\beta X|^+).$ By the continuity of $\bar{f},$ we have that $$\bar{f}\left[\{y\}\times \overline{A_y}^{T(|\beta X|^+)}\right]=f(x_y,\alpha_y) \mbox{ holds for all } y \in C_2.$$

																																			      \vskip 15pt
																																			      We'll first prove that $C_1\cap \overline{C_2}^{\beta X}=\emptyset$. 
																																			      Suppose the contrary. Let $x\in C_1\cap \overline{C_2}^{\beta X}$ and $T_x$ as defined in \textbf{Lemma 4.3}. 
																																			      Let $T=\bigcap \left\{ \overline{A_y}^{T(|\beta X|^+)} : y\in C_2 \right\}  \cap T_x,$. Now, by \textbf{Fact 3.6}, $|T|=\left|\beta X\right|^+ $, and $T$ is closed and unbounded in $T(\left|\beta X \right| ^+)$. \\

																																			      \vskip 10pt

																																			      As $x\in C_1$ and $T\subseteq T_x$, so we must have $\bar{f}\left[\{x\} \times T \right] = f\left[\{y_x\}\times T\right].$ Since $f$ is one-to-one, $\left|f\left[\{y_x\}\times T\right]\right|=|T|=\left|\beta X \right| ^+$. So, $\left|\bar{f}\left[\{x\}\times T\right]\cap Z\right|=|\beta X|^+.$

																																			      \vskip 5pt
																																			      On the other hand, since $x\in \overline{C_2}^{\beta X},$ we have $\left|\bar{f}\left[\{x\} \times T \right] \cap Z \right| < \left|\beta X \right|^+$ holds. Because: \\
																																				  $$\bar{f}\left[\{x\}\times T\right] \cap Z 
																																				  \subseteq \bar{f}\left[\overline{C_2}^{\beta X} \times T\right] \cap Z 
																																				  \subseteq \bar{f}\left[\overline{C_2\times T}^{\beta X \times T(|\beta X|^+ +1)}\right] \cap Z$$
																																				  $$=\bar{f}\left[\overline{C_2\times T}^{\beta \left(X \times T(|\beta X|^+ )\right)}\right] \cap Z
																																				  \subseteq \overline{\bar{f}\left[C_2\times T\right]}^{\beta Z} \cap Z
																																				  =\overline{    \left\{ f(x_y,\alpha_y): y\in C_2 \right\}        }^{\beta Z},$$

																																				  where the last equality holds because for each $y\in C_2$, $\bar{f}\left[\{y\}\times T \right]=\{f(x_y,\alpha_y)\}$ for some $(x_y,\alpha_y)\in X\times T(|\beta X|^+).$ Now, let $\alpha=\sup\{\alpha_y: y\in C_2\}$, which exists because $|C_2|< cf\left( \left|\beta X\right|^+ \right)$. 

																																				  Since $\left\{ f(x_y,\alpha_y): y\in C_2 \right\} 
																																				  \subseteq \bar{f}\left[\beta X \times T(\alpha +1) \right] = \overline{ \bar{f}\left[\beta X \times T(\alpha +1) \right]}^{\beta Z}$, we must have $\overline{    \left\{ f(x_y,\alpha_y): y\in C_2 \right\}        }^{\beta Z} \subseteq \bar{f}\left[\beta X \times T(\alpha +1) \right] $. 

																																				  Hence, $\bar{f}\left[\{x\}\times T\right] \cap Z \subseteq \bar{f}\left[\beta X \times T(\alpha +1) \right].$ 
																																				  As $\left| \bar{f}\left[\beta X \times T(\alpha +1) \right] \right| < |\beta X |^+$, we conclude that 
																																				  $\left|\bar{f}\left[\{x\} \times T \right] \cap Z \right| < \left|\beta X \right|^+$. This contradicts to our 
																																				  previous argument that $ \left|\bar{f}\left[\{x\} \times T \right] \cap Z \right| = \left|\beta X \right|^+$. Therefore, 
																																				  $C_1\cap \overline{C_2}^{\beta X}=\emptyset$. 


																																				  \vskip 10pt

																																				  We will show that $X\cap \overline{C_2}^{\beta X}=\emptyset$. Suppose the contrary. Let $x\in X \cap \overline{C_2}^{\beta X}$. By replacing $T$ with $T'=\bigcap \left\{ \overline{A_y}^{T(|\beta X|^+)} : y\in C_2 \right\}$ in the above, same conclusion follows, namely  $\left|\bar{f}\left[\{x\} \times T \right] \cap Z \right| < \left|\beta X \right|^+$. \\

																																				  However, since $x\in X$, $\left|\bar{f}\left[\{x\} \times T \right] \cap Z \right| =\left| f\left[\{x\} \times T \right]\cap Z \right| = \left| f\left[\{x\} \times T\right]\right|= \left|\beta X \right|^+$, where the last equality holds because $f$ is one-to-one and $|T|=|\beta X|^+$. 
																																				  Again, we reached contradiction. Thus, $X\cap \overline{C_2}^{\beta X}=\emptyset$ holds. \qed



																																				  \vskip 40pt


																																				  We will now prove the theorem, one direction is trivial. Buzjakova proved the hard direction in 1997. We consider only Tychonoff spaces. 
																																				  \vskip 20pt
																																				  \textbf{Buzjakova's Theorem.} A pseudocompact space $X$ condenses onto a compact space if and only if the space $X\times T(\left| \beta X \right| ^+ +1)$ condenses onto a normal space. 
																																				  \vskip 15pt

																																				  \textbf{Proof.}\vskip 10pt
																																				  \textbf{($\Rightarrow$:) } Let $X$ be a pseudocompact space that condenses onto a compact space $K$. So there exists $f:X\rightarrow K$ such that $f$ is one-to-one, onto, and continuous. Define $g: X\times T(|\beta X|^+ +1) \rightarrow K\times T(|\beta X|^+ +1)$ by $g(x,\alpha)=(f(x), \alpha).$ Then, $g$ is one-to-one, onto, and continuous. Hence $X\times T(|\beta X|^+ +1)$ condenses onto a normal space. 


																																				  \vskip 20pt

																																				  \textbf{($\Leftarrow$:)} By \textbf{Fact 3.7}, the space $X\times T(|\beta X|^+)$ condenses onto $X\times T(|\beta X|^+ +1).$ 
																																				  Since $X\times T(|\beta X|^+ +1)$ condenses onto some normal space $Z$ by our assumption, the space $X\times T(|\beta X|^+)$ must condenses onto $Z$, too. 
																																				  So, there exists $f: X\times T(|\beta X|^+) \rightarrow Z$, where $f$ is ono-to-one, onto, and continuous. 
																																				  Let \vskip 10pt
																																				  $C_1=\left\{x\in  \beta X \backslash X: |\bar{f}[\{x\} \times T(|\beta X|^+)] \cap Z|=|\beta X|^+\right\}$, and \vskip 10pt
																																				  $C_2=\{x\in \beta X\backslash X: \left|\bar{f}[\{x\}\times T(|\beta X|^+)]\cap Z\right|<|\beta X|^+ \mbox{ and for any }\alpha\in T(|\beta X|^+),$\\ $\mbox{ there exists }\alpha_1\in T(|\beta X|^+)\mbox{ such that }\alpha_1 >\alpha \mbox{ and } \bar{f}(x,\alpha_1) \in Z \}.$

																																				  \vskip 15pt
																																				  Let $T=\bigcap\{T_x:x\in C_1\}$, where $T_x$ was defined in \textbf{Lemma 4.3}. By \textbf{Fact 3.6,} $T$ is closed and unbounded in $T(|\beta X|^+)$ and $|T|=|\beta X|^+.$

																																				  \vskip 25pt

																																				  Let $x\in \overline{C_2}^{\beta X}.$ Since $\overline{C_2}^{\beta X}\cap (C_1\cup X)=\emptyset$ by \textbf{Lemma 4.4}, $x\in \beta X \backslash (C_1\cup X)$. Thus, $$\left|\bar{f}\left[ \{x\}\times T(|\beta X|^+) \right] \cap Z \right| < |\beta X|^+.$$ 
																																				  Let $\alpha_x=\sup\left\{ \alpha \in T(|\beta X|^+): f(y,\alpha)\in \bar{f}\left[ \{x\}\times T(|\beta X|^+) \right] \cap Z \mbox{ for some } y\in X\right\}$. Since   $\left|\bar{f}\left[ \{x\}\times T(|\beta X|^+) \right] \cap Z \right| < |\beta X|^+,$ such $\alpha_x$ exists in $T(|\beta X|^+)$. 
																																				  For any $\alpha_1 > \alpha_x$, we have either $\bar{f}(x,\alpha_1)\in \beta Z \backslash Z$ or $\bar{f}(x,\alpha_1)\in Z$. If $\bar{f}(x,\alpha_1)\in Z,$ we must have $\bar{f}(x,\alpha_1)=f(y,\alpha_2)$ for some $\alpha_2$ because $f$ is onto. Furthermore, since $\alpha_1>\alpha_x$ and by our definition of $\alpha_x, \alpha_1>\alpha_2$ must hold.
																																				  \vskip 15pt

																																				  Let $\alpha^* = \sup \left\{\alpha_x: x\in \overline{C_2}^{\beta X}\right\}.$ Such $\alpha^*$ exists because $\left| \overline{C_2}^{\beta X} \right| <cf(|\beta X|^+).$  For any $\alpha>\alpha^*, \alpha \in T,$
																																				  $$f\left[X\times \{\alpha\}\right] \cap \bar{f}\left[\overline{C_2}^{\beta X} \times \{\alpha\}\right] = \emptyset$$ holds. 

																																				  This is because if $z\in f\left[X\times \{\alpha\}\right] \cap \bar{f} \left[\overline{C_2}^{\beta X} \times \{\alpha\} \right]$, we can write $z=\bar{f}(x,\alpha)$ for some $x\in \overline{C_2}^{\beta X}$. From the definition of $\alpha^*$ and the fact that $\alpha>\alpha^*$, we have either $\bar{f}(x,\alpha)=f(y,\alpha_2)$ where $y\in X$ and $\alpha_2<\alpha,$ or, we have $\bar{f}(x,\alpha)\in \beta Z\backslash Z.$ Suppose that $\bar{f}(x,\alpha)=f(y,\alpha_2)$ holds. Since $f$ is one-to-one, $\bar{f}(x,\alpha)=f(y,\alpha_2)\neq f(y',\alpha)$ for any $y'\in X.$ Thus, $z\notin f\left[X\times \{\alpha\}\right] $, a contradiction. If $\bar{f}(x,\alpha)\in \beta Z\backslash Z$ holds, $\bar{f} (x,\alpha)\notin Z \supseteq f\left[X\times \{\alpha\}\right]$, a contradiction. Thus, $f\left[X\times \{\alpha\}\right] $ and $\bar{f}\left[\overline{C_2}^{\beta X} \times \{\alpha\}\right]$ are disjoint.

																																				  \vskip 25pt


																																				  Having defined $\alpha^*$, we have two cases: 

																																				  \vskip 20pt

																																				  \textbf{CASE I.} For all $\alpha \in T, \alpha>\alpha^*,$ the set $f\left[X\times\{\alpha\}\right] \cup \bar{f}\left[\overline{C_2}^{\beta X} \times \{\alpha\}\right] $is not compact. 

																																				  \vskip 20pt

																																				  Let $$C_3=(\beta X\backslash X) \backslash (C_1\cup \overline{C_2}^{\beta X}).$$

																																				  Note that $C_3$ is nonempty because if it is , then $\beta X\backslash X= C_1\cup \overline{C_2}^{\beta X}.$ We will show that $f\left[\beta X\times \{\alpha\} \right] \subseteq f\left[X\times\{\alpha\}\right]\cup\bar{f}\left[\overline{C_2}^{\beta X}\times\{\alpha\}\right].$ Let $x\in\beta X\backslash X,$ if $x\in C_1$, then $\bar{f}(x,\alpha)=f(y_x,\alpha)$ for some $y_x\in X$ because $\alpha\in T\subseteq T_x.$ Hence $\bar{f}(x,\alpha)\in f\left[X\times\{\alpha\}\right].$ If $x\in \overline{C_2}^{\beta X}$, then $\bar{f}(x,\alpha)\in \bar{f}\left[\overline{C_2}^{\beta X}\times \{\alpha\}\right].$ 
																																				  The other inclusion is trivial. So now, we have 
																																				  $$\bar{f}\left[\beta X\times \{\alpha\}\right]=f\left[X\times\{\alpha\}\right]\cup\bar{f}\left[\overline{C_2}^{\beta X} \times \{\alpha\}\right].$$
																																				  However, as $\bar{f}$ is continuous and $\beta X\times \{\alpha\}$ is compact, $\bar{f}\left[\beta X\times \{\alpha\}\right]$ is compact. But by our assumption, $f\left[X\times\{\alpha\}\right] \cup \bar{f}\left[\overline{C_2}^{\beta X} \times \{\alpha\}\right] $ is not compact, a contradiction. Hence, $C_3$ is nonempty.

																																				  \vskip 20pt


																																				  For each $x\in C_3,$ there exist $\alpha_x\in T(|\beta X|^+)$ such that for $\alpha>\alpha_x$, $\bar{f}(x,\alpha)\in \beta Z\backslash Z$ holds. This follows from the definition of $C_1$ and $C_2$. Let $\beta^*=\sup\{\alpha_x: x\in C_3\}.$ Such $\beta^*$ exists because $|C_3|<cf(|\beta X|^+).$

																																				  Let $$\gamma^*=\max\{\alpha^*, \beta^*\}.$$

																																				  \vskip 15pt
																																				  Fix an arbitrary $\lambda \in T, \lambda >\gamma^*.$
																																				  \vskip 10pt
																																				  The set $f\left[X\times \{\lambda\}\right]$ is closed in $Z$. Suppose not, by \textbf{Lemma 4.2}, there exists 
																																				  $$(x,\lambda)\in \overline{X\times \{\lambda\}}^{\beta X\times T(|\beta X|^+ +1)} \backslash X\times \{\lambda\}$$ such that $$\bar{f}(x,\lambda) \in \overline{f\left[X\times\{\lambda\}\right]}^Z \backslash f\left[X\times \{\lambda\}\right].$$ 
																																				  \\
																																				      Since $(x,\lambda)\notin X\times \{\lambda\}, x\in \beta X\backslash X.$ Note that the remainder $\beta X\backslash X$ is partitioned into the sets $C_1, \overline{C_2}^{\beta X},$ and $C_3$. If $x\in C_1,$ then since $\lambda \in T\subseteq T_x$, there exists $y_x\in X$ such that $\bar{f}(x,\lambda)= f(y_x,\lambda) \in f\left[X\times\{\lambda\}\right],$ a contradiction. If $x\in \overline{C_2}^{\beta X},$ then either $\bar{f}(x,\lambda)=f(y,\lambda_2)$ and $\lambda>\lambda_2$, or $\bar{f}(x,\lambda)\in \beta Z\backslash Z$. We reach contradiction in both cases. Finally, if $x\in C_3,$ then $\bar{f}(x,\lambda)\in \beta Z\backslash Z$, contradiction again. Therefore, $f\left[X\times\{\lambda\}\right]$ is closed in $Z$.


																																				      \vskip 20pt

																																				      So far, we have that $f\left[X\times\{\lambda\}\right]$ is closed in $Z$, $\bar{f}\left[\overline{C_2}^{\beta X} \times \{\alpha\}\right]$ is compact in $\beta Z$, $f\left[X\times \{\alpha\}\right]\cap \bar{f}\left[\overline{C_2}^{\beta X} \times \{\alpha\}\right] = \emptyset$, and $f\left[X\times \{\alpha\}\right]\cup \bar{f}\left[\overline{C_2}^{\beta X} \times \{\alpha\}\right]$ is not compact in $\beta Z$. Thus, by \textbf{Fact 3.8,} there exists a system $D=\{D_\alpha:\alpha<l\}\subseteq \mathcal{P}\left(f\left[X\times\{\lambda\}\right]\right)$ satisfying these 4 conditions:

																																				      \begin{enumerate}
																																				      \item For each $\alpha$, the set $D_\alpha$ is nonempty and closed in $f\left[X\times \{\lambda\}\right].$
																																				      \item For $\alpha>\beta, D_\alpha\subseteq D_\beta,$ and if $\beta$ is a limit ordinal, then $D_\beta=\bigcap\left\{ D_\alpha: \alpha< \beta\right\}$.
																																				      \item $\bigcap \{D_\alpha:\alpha<l\}=\emptyset$.
																																				      \item $\overline{D_1}^{\beta Z} \cap \bar{f}\left[\overline{C_2}^{\beta X} \times \{\lambda\}\right]=\emptyset.$

																																				      \vskip 30pt


																																				      Since $f$ is one-to-one and $D_\alpha\subseteq f\left[X\times \{\lambda\}\right],$ we have $ f^{\leftarrow}\left[D_\alpha\right]\subseteq X\times\{\lambda\}$ for any $\alpha$. Define a system $A=\{A_\alpha\}$ of subsets of $X$ such that $f^{\leftarrow}\left[D_\alpha\right]=A_\alpha\times\{\lambda\}.$ We will show that $A$ satisfies the following conditions: 


																																				      \vskip 25pt


																																				      \item For each $\alpha$, the set $A_\alpha$ is closed in $X$.\\
																																					  \texttt{Proof-} Since $D_\alpha$ is closed in $f\left[X\times\{\lambda\}\right]$ and $f\left[X\times\{\lambda\}\right]$ is closed in $Z$, $D_\alpha$ must be closed $Z$ as well. As $f$ is continuous, $f^{\leftarrow}\left[D_\alpha\right]$ is closed in $X\times T(|\beta X|^+).$ Since $f^{\leftarrow}\left[D_\alpha\right]=A_\alpha \times \{\lambda\},$  $A_\alpha$ is closed in $X$. \


																																					  \vskip 20pt	


																																					  \item $\bigcap\{A_\alpha\}=\emptyset.$\\
																																					      \texttt{Proof-} By our definition of $A_\alpha$, we have $f^{\leftarrow}\left[\bigcap\{D_\alpha\}\right]=\bigcap\left\{f^{\leftarrow}\left[D_\alpha\right]\right\}=\bigcap \{A_\alpha\}\times\{\lambda\}.$ Since $\bigcap \{D_\alpha\}=\emptyset $ by condition 3, then $\bigcap \{A_\alpha\}\times\{\lambda\}= \emptyset$, and thus $\bigcap \{A_\alpha\}= \emptyset.$


																																					      \vskip 20pt	

																																					      \item For $\alpha>\beta, A_\alpha \subseteq A_\beta$ and if $\beta$ is a limit ordinal, then $A_\beta=\bigcap\{A_\alpha\}.$\\
																																																		    \texttt{Proof-} If $\alpha>\beta, D_\alpha\subseteq D_\beta$ holds by condition 2. As $f^{\leftarrow}\left[D_\alpha\right]\subset f^{\leftarrow}\left[D_\beta\right], A_\alpha\times \{\lambda\} \subseteq A_\beta\times \{\lambda\}$. Thus, $A_\lambda \subseteq A_\beta.$ Now, if $\beta$ is a limit ordinal, then by condition 2, $D_\beta=\bigcap \{D_\alpha\}.$ So $A_\beta\times\{\lambda\}=f^{\leftarrow}\left[D_\beta\right]=f^{\leftarrow}\left[\bigcap\{D_\alpha\}\right]=\bigcap f^{\leftarrow}\left[D_\alpha\right]=\bigcap \left\{A_\alpha\times \{\lambda\}\right\}.$ Thus, $A_\beta=\bigcap \{A_\alpha\}.$



																																																		    \vskip 20pt


																																																		    \item $\overline{A_1}^{\beta X} \cap \overline{C_2}^{\beta X}=\emptyset.$ \\
																																																			\texttt{Proof-} Suppose there exists $x\in \overline{A_1}^{\beta X}\cap \overline{C_2}^{\beta X}.$ Since $x\in \overline{A_1}^{\beta X}, \bar{f}(x,\lambda)\in \bar{f}\left[\overline{A_1}^{\beta X}\times \{\lambda\}\right] = \bar{f}\left[\overline{A_1 \times \{\lambda\}}^{\beta X\times T(|\beta X|^+ +1)}\right]=\bar{f}\left[\overline{A_1 \times \{\lambda\}}^{\beta (X\times T(|\beta X|^+ ))}\right]\subseteq \overline{\bar{f}\left[A_1\times\{\lambda\}\right]}^{\beta Z}=\overline{f\left[A_1\times\{\lambda\}\right]}^{\beta Z}=\overline{D_1}^{\beta Z}$. On the other hand, since $x\in \overline{C_2}^{\beta X},$ $\bar{f}(x,\lambda)\in \bar{f}\left[\overline{C_2}^{\beta X}\times \{\lambda\}\right].$ Therefore, $\bar{f}(x,\lambda)\in \overline{D_1}^{\beta Z} \cap \bar{f}\left[\overline{C_2}^{\beta X}\times \{\lambda\}\right].$ However, by condition 4,  $\overline{D_1}^{\beta Z} \cap \bar{f}\left[\overline{C_2}^{\beta X}\times \{\lambda\}\right]=\emptyset$, a contradiction. So, $\overline{A_1}^{\beta X} \cap \overline{C_2}^{\beta X}=\emptyset.$


																																																			\vskip 20pt

																																																			\item If $x\in \bigcap \overline{A_\alpha}^{\beta X}$, then $x\in C_3.$\\
																																																			    \texttt{Proof-} Since $\beta X\backslash X$ is partitioned in to $C_1, \overline{C_2}^{\beta X},$ and $C_3$, we only need to show that $x\notin C_1$ and $x\notin \overline{C_2}^{\beta X}.$ From condition 8, $\overline{A_1}^{\beta X} \cap \overline{C_2}^{\beta X}=\emptyset,$ so $x\notin \overline{C_2}^{\beta X}.$  Now suppose that $x\in C_1$, then since $\lambda \in T \subseteq T_x$, $\bar{f}(x,\lambda)=f(y_x,\lambda)$ for some $y_x\in X$. Since $x\in \bigcap \overline{A_\alpha}^{\beta X}$, we have $\bar{f}(x,\lambda)\in \bar{f}\left[\overline{A_\alpha}^{\beta X}\times \{\lambda\}\right]$ for all $\alpha$. Thus, $f(y_x,\lambda)\in \bar{f}\left[\overline{A_\alpha}^{\beta X}\times \{\lambda\}\right]=\\ \bar{f}\left[\overline{A_\alpha \times \{\lambda\}}^{\beta X\times T(|\beta X|^+ +1)}\right]=\bar{f}\left[\overline{A_\alpha \times \{\lambda\}}^{\beta (X\times T(|\beta X|^+ ))}\right]\subseteq \overline{\bar{f}\left[A_\alpha \times \{\lambda\}\right]}^{\beta Z}\\
																																																			    =\overline{f\left[A_\alpha \times \{\lambda\}\right]}^{\beta Z}.$	              
																																																			    Since $f(y_x,\lambda)\in Z$, and by condition 10, $f\left[A_\alpha\times\{\lambda\}\right]$ is closed in $Z$, we must have $f(y_x,\lambda)\in f\left[A_\alpha\times\{\lambda\}\right]$. As $f$ is one-to-one, $y_x\in A_\alpha$ for each $\alpha$. However, that means $y_x\in \bigcap \{A_\alpha\}$, contradicting condition 6, which says $\bigcap \{A_\alpha\}=\emptyset$. So, $x\notin C_1$, and therefore $x$ must be in $C_3$. 


																																																			    \vskip 20pt


																																																			    \item The set $f\left[A_\alpha\times\{\gamma\}\right]$ is closed in $Z$ for each $\gamma>\gamma^*, \gamma\in T$.\\
																																																				\texttt{Proof-} By \textbf{Lemma 4.2,} there exists $x\in \overline{A_\alpha}^{\beta X}\backslash A_\alpha$ such that $\bar{f}(x,\gamma)\in \overline{f\left[A_\alpha\times\{\gamma\}\right]}^Z \backslash f\left[A_\alpha\times \{\gamma\}\right]$ holds. By condition 8, $\overline{A_1}^{\beta X} \cap \overline{C_2}^{\beta X}=\emptyset, \mbox{ so }x\notin \overline{C_2}^{\beta X}.$ Now, if $x\in C_3,$ then $\bar{f}(x,\gamma)=f(x,\gamma')$ for some $\gamma'>\gamma^*$, a contradiction, so $x\notin C_3$. Thus, $x\in C_1.$ Since $\lambda \in T\subseteq T_x,$ $\bar{f}(x,\lambda)=f(y_x,\lambda)$ for some $y_x\in X$, and so $\bar{f}(x,\lambda)\in Z$. As $f\left[A_\alpha\times\{\lambda\}\right]=D_\alpha$ is closed in $Z$ and $\bar{f}(x,\lambda)\in Z$, $\bar{f}(x,\lambda)\in f\left[A_\alpha \times \{\lambda\}\right].$ Moreover, since $\bar{f}(x,\lambda)=f(y_x, \lambda), f(x,\lambda)\in f\left[A_\alpha \times \{\lambda\}\right].$ As $f$ is one-to-one, $y_x\in A_\alpha$. Note that $y_x\in A_\alpha$ is fixed, and by \textbf{Lemma 4.3}, for all $\gamma \in T\subseteq T_x,$ $\bar{f}(x,\gamma)=f(y_x,\gamma)\in f\left[A_\alpha \times \{\gamma\}\right].$ We reach a contradiction.

																																																				\end{enumerate}



																																																				\vskip 30pt


																																																				Let $\gamma=|A|$. Note that $|A|\leq |X|<|\beta X|^+$ because $A=\left\{A_\alpha: \alpha< l\right\} \subseteq \mathcal{P}(X)$ and $\left\{A_\alpha\right\}$ is decreasing. 
																																																				Choose a closed subset $G=\{\gamma_\alpha: \alpha\leq \gamma\}$ of $T$ such that $\gamma_1>\gamma^*$ and for $\alpha>\beta, \gamma_\alpha>\gamma_\beta$. Define 
																																																				$$B_1=\bigcup_{\alpha<\gamma} \left\{A_\alpha\times \{\gamma_\alpha\}\right\}$$
																																																				$$B_2=A_1\times \left\{\gamma_{\gamma}\right\}$$ \vskip 5pt

																																																				$B_1$ is closed in $X\times T(|\beta X|^+)$ because $\bigcap \{A_\alpha\}=\emptyset$ by condition 6 and our choice of $G$. $B_2$ is closed in $X\times T(|\beta X|^+)$ because $A_1$ is closed by condition 5. $B_1$ and $B_2$ are disjoint because $B_1$ doesn't contain any element with the second coordinate equal to $\gamma_\gamma$.

																																																				\vskip 15pt

																																																				Now, as each $\overline{A_\alpha}^{\beta X} \neq \emptyset$, there exists $x\in \bigcap \left\{\overline{A_\alpha}^{\beta X}\right\}.$ Thus $(x,\gamma_\gamma)\in \overline{B_1}^{\beta X \times T(|\beta X|^+ +1)}$. Moreover, $(x, \gamma_\gamma)\in \overline{A_1}^{\beta X} \times \left\{\gamma_\gamma\right\}= \overline{B_2}^{\beta X \times T(|\beta X|^+ +1)}$. Thus, $(x,\gamma_\gamma)\in \overline{B_1}^{\beta X\times T(|\beta X|^+ +1)} \cap \overline{B_2}^{\beta X\times T(|\beta X|^+ +1)}=\overline{B_1}^{\beta\left( X\times T(|\beta X|^+)\right)} \cap \overline{B_2}^{\beta \left(X\times T(|\beta X|^+)\right)}.$ By \textbf{Fact 3.9}, the sets $B_1$ and $B_2$ are not completely separated in $X\times T(|\beta X|^+).$

																																																				\vskip 15pt


																																																				Let us consider the $f\left[B_1\right]$ and $f\left[B_2\right].$ Since $f$ is one-to-one, we have $f\left[B_1\right] \cap f\left[B_2\right]=\emptyset.$ 
																																																				By condition 10, $f\left[B_2\right]$ is closed in $Z$. We shall prove that $f\left[B_1\right]$ is closed in $Z$. Assume the contrary. Then, by \textbf{Lemma 4.2}, there exists $(x,\gamma_\alpha)\in \overline{B_1}^{\beta X\times T(|\beta X|^+ +1)} \backslash B_1$ such that $\bar{f}(x,\gamma_\alpha)\in \overline{f\left[B_1\right]}^Z\backslash f\left[B_1\right].$ 
																																																				\vskip 15pt

																																																				If $\alpha$ is isolated, since $(x,\gamma_\alpha)\in \overline{B_1}^{\beta X\times T(|\beta X|^+ +1)}, (x,\gamma_\alpha)\in \overline{A_\alpha\times \{\gamma_\alpha\}}^{\beta X \times T(|\beta X|^+ +1)}.$ So $\bar{f}(x,\gamma_\alpha)\in \bar{f}\left[\overline{A_\alpha\times \{\gamma_\alpha\}}^{\beta X\times T(|\beta X|^+ +1)}\right]=\bar{f} \left[\overline{A_\alpha\times \{\gamma_\alpha\}}^{\beta (X\times T(|\beta X|^+)}\right] \subseteq \overline{\bar{f}\left[A_\alpha\times \{\gamma_\alpha\}\right]}^{\beta Z}=\overline{f\left[A_\alpha\times \{\gamma_\alpha\}\right]}^{\beta Z}.$  If $\bar{f}(x,\gamma_\alpha)\notin Z$, we reach a contradiction. If $\bar{f}(x,\gamma_\alpha)\in Z$, then $\bar{f}(x,\gamma_\alpha)\in
																																																					\overline{f\left[A_\alpha\times \{\gamma_\alpha\}\right]}^{\beta Z}\cap Z=f\left[A_\alpha \times \{\gamma_\alpha\}\right],$ where the last equality holds because by condition 10, $f\left[A_\alpha\times \{\gamma_\alpha\}\right]$ is closed in $Z$. Now, $\bar{f}(x,\gamma_\alpha)\in f\left[A_\alpha \times \{\gamma_\alpha\}\right]\subseteq f\left[B_1\right]$ is a contradiction.
																																																					\vskip 15pt
																																																					If $\alpha$ is a limit ordinal not equal to $\gamma$, then by condition 7 and that $(x,\gamma_\alpha)\in \overline{B_1}^{\beta X\times T(|\beta X|^+ +1)},$  we have $(x,\gamma_\alpha)\in \overline{A_\alpha\times \{\gamma_\alpha\}}^{\beta X \times T(|\beta X|^+ +1)}.$ Then by the same reasoning as above, we reach a contradiction.
																																																					\vskip 15pt
																																																					Now, if $\alpha=\gamma$, then since $(x,\gamma_\gamma) \in \overline{B_1}^{\beta \times T(|\beta X|^+ +1)}\backslash B_1$, and by the way $B_1$ was defined, we must have $x\in \bigcap \left\{\overline{A_\alpha}^{\beta X}\right\}$. By condition 9, $x\in C_3$. But since $\gamma_\gamma> \gamma^* \geq \beta^*, \bar{f}(x,\gamma_\gamma)\in \beta Z\backslash Z,$ a contradiction. Hence, the set $f[B_1]$ must be closed in $Z$.

																																																					\vskip 25pt

																																																					Now, the sets $f\left[B_1\right]$ and $f\left[B_2\right]$ are closed and disjoint in $Z$. Since $Z$ is normal, by Urysohn's Lemma, there exists a continuous function $g:Z\rightarrow [0,1]$ such that $g\left[ f\left[B_1\right]\right]\subseteq \{0\}$ and $g\left[f\left[B_2\right]\right]\subseteq \{1\}.$ Let $h=g\circ f.$ Then $B_1$ and $B_2$ are completely separated by the continuous function $h$. This is a contradiction, and so \textbf{CASE I} can never happen.  















																																																				\vskip 30pt

																																																				\textbf{CASE II.} There exists an ordinal $\alpha \in T, \alpha>\alpha^*$ such that the set $Y=f\left[X\times\{\alpha\}\right] \cup \bar{f}\left[\overline{C_2}^{\beta X} \times \{\alpha\}\right] $is compact. 

																																																				\vskip 10pt

																																																				Since $f\left[X\times\{\alpha\}\right] \cap \bar{f}\left[\overline{C_2}^{\beta X} \times \{\alpha\}\right]=\emptyset$ and the set 
																																																				$\bar{f}\left[\overline{C_2}^{\beta X} \times \{\alpha\}\right]$ is compact, then $f\left[X\times \{\alpha\}\right]$ is relatively open in the compact space $Y$; so  $f\left[X\times\{\alpha\}\right]$ is locally compact.
																																																				\vskip 10pt

																																																				By \textbf{Fact 3.10}, $f\left[X\times\{\alpha\}\right]$ condenses onto a compact space. Hence, $X\times\{\alpha\}$ condenses onto a compact space as well. Since $X$ is homeomorphic to $X\times \{\alpha\}, X$ condenses onto a compact space, too. The theorem is proved. \qed



																																																				\newpage


																																																				\begin{center}
																																																				\textbf{References}
																																																				\end{center}

																																																				\vskip 30pt

																																																				\begin{itemize}
																																																				\item Buzjakova, R., A criterion that a pseudocompact space condenses onto a compact space, Q\& A in General Topology, Vol. 15, 1997.
																																																				\item Engelking, R., General Topology, Heldermann Verlad Berlin, 1989.
																																																				\item Porter, J. and R. G. Woods, Extensions and absolutes of Hausdorff Spaces, Springer-Verlag, 1988.
																																																				\item Tamano, H., On Paracompactness, Pacific Journal of Mathematics, Vol. 10, 1960. 
																																																				\item Willard, S., General Topology, Dover Publications Inc, 1970.
																																																				\end{itemize}














																																																				\end{document}
