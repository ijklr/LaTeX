\documentclass{article}
\begin{document}

\textbf{FACT 5.} Let $Z$ be a Tychonoff space. Let $A$ be a closed subset of $Z$, $B$ be compact subset of $\beta Z$. 
The set $A\cup B$ is not compact in $\beta Z$. 
Then, there exists a system $D=\{D_\alpha\}$ satisfying the following conditions:
\begin{enumerate}
\item For each $  \alpha  $, the set $ D_\alpha  $ is non-empty and closed in $A$.
\item For $ \alpha>\beta, D_\alpha \subseteq D_\beta  $ and if $  \beta $ is a limit ordinal number, then\\$  D_\beta=\bigcap \{D_\alpha:\alpha <\beta\}.$
\item $ \bigcap\{D_\alpha\}=\emptyset . $
\item $(\overline{D_1}^{\beta Z}\cap B =\emptyset.$
	 
\end{enumerate}

\textbf{Proof.} \\
Since $A\cup B $ is not compact, there is an open cover $\mathcal{C} \subset \tau(\beta Z)$ of $A\cup B$ that has no finite subcover.  Since $B$ is compact and $\mathcal{C}$ covers $B$, $\mathcal{C}$ yields a finite subcover $\{C_i:1\leq i \leq n\}\subseteq \mathcal{C}$.
\vskip 10pt

Let $E=A\backslash \bigcup \{C_i:1\leq i \leq n\}$. $E$ is closed in $A$ because $\bigcup \{C_i:1\leq i \leq n\}$ is closed in $\beta Z$. $E$ is nonempty because if it is, then that means $\{C_i:1\leq i \leq n\}$ covers $A$ as well as $B$, contradiction. Furthermore, $E$ is not compact. If $E$ is compact, we can get a finite subcover $\{C'_i:1\leq i\leq n\}$ from $\mathcal{C}$. Then, $\{C'_i:1\leq i\leq n\} \cup \{C_i:1\leq i \leq n\}$ is a finite subcover that covers $A\cup B$, contradiction. 


\vskip 10pt


As $E$ is not compact, we can find an open cover $\mathcal{F}\subseteq \tau(\beta Z)$ such that no finite subcover of $\mathcal{F}$ covers $E$. 
WLOG, we can assume that $|\mathcal{F}|=L(E)$, the Lindeloff number of $E$. We can well-order $\mathcal{F}$, so $\mathcal{F}=\left\{F_\alpha: \alpha<L(E)\right\}$.
Define $D_\alpha=E\backslash \bigcup \left\{F_\gamma: \gamma<\alpha\right\}$. 


\begin{enumerate}
\item For each $  \alpha  $, the set $ D_\alpha  $ is non-empty and closed in $A$.\\
\texttt{Proof- Each $D_\alpha$ is non-empty because if $D_\alpha=\emptyset$ for some $\alpha$, then since $\alpha<L(E)$, and $E$ can be covered by $\{F_\gamma: \gamma<\alpha\}$, smaller than its Lindeloff number, we have a contradiction. $D_\alpha$ is closed by the way we defined them.}

\item For $ \alpha>\beta, D_\alpha \subseteq D_\beta  $ and if $  \beta $ is a limit ordinal number, then\\$  D_\beta=\bigcap \{D_\alpha:\alpha <\beta\}.$\\
\texttt{Proof- By the way we defined the $D_\alpha's$, $D_\alpha\subseteq D_\beta$ if $\alpha>\beta$. If $\beta$ a limit ordinal number, and 
if $D_\beta \neq \bigcap\left\{D_\alpha:\alpha<\beta\right\}, $ then we replace $D_\beta$ with the set $\bigcap\left\{D_\alpha:\alpha<\beta\right\}$ is closed in $A$. This new $D_\beta$ is closed in $A$.}

\item $ \bigcap\{D_\alpha\}=\emptyset . $\\
\texttt{Proof- This is true because $\mathcal{F}$ covers $A$.}


\item $\overline{E}^{\beta Z} \cap B =\emptyset.$\\
\texttt{Since $\overline{E}^{\beta Z} \cap B=\emptyset$, and $D_1\subseteq E$, then $\overline{D_1}^{\beta Z}\cap B=\emptyset.$}
 
\end{enumerate}







\end{document}