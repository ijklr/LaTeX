\documentclass{article}
\begin{document}
\textbf{FACT 4.} If $\tau>\omega$ is an infinite cardinal, then $T(\tau)$ can be condensed onto $T(\tau+1).$ Moreover, for any space $X, X\times T(\tau)$ condenses onto $X\times T(\tau+1).$

\vskip 15pt

\textbf{Proof.} Define $g:T(|\beta X|^+) \rightarrow T(|\beta X|^+)$ by $g(0)=\tau$ and $g(\alpha)=\alpha-1$ for all $\alpha<\omega$. Now, $g$ is one-to-one and onto. Also, $g$ is continuous at $\omega$ because if $(\beta, \omega]$ is an open set containing $g(\omega)$,then $(\beta+1,\omega]$ is an open set such that $g\left[(\beta+1, \omega]\right] \subseteq (\beta, \omega];$ $g$ is continuous on all $\alpha<\omega$ because $\{\alpha\} \in T(\tau)$; finally, $g$ is continuous on all $\alpha>\omega$ because $g|_{\alpha<\omega}$ is an identity function. 

\vskip 10pt


Moreover, dfine $h: X\times T(\tau)\rightarrow X\times T(\tau+1)$ by $h(x,\alpha)=(x,g(\alpha)).$ Since $g$ is one-to-one, onto, and continuous, hence $h$ must also be one-to-one, onto, and continuous. 










\end{document}